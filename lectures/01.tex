\section{Элементарная теория вероятностей.}
\subsection{Классическая модель теории вероятностей.}

\begin{df*}
 Пусть $\Omega = \left\{ \omega_1, \ldots, \omega_n \right\}$ --- конечное множество, называемое \textit{пространством элементарных исходов,} если
 \begin{itemize}
  \item $\omega_i$ равновозможны;
  \item $\omega_i$ не совместны;
  \item одно из $\omega_i$ всегда реализуется.
 \end{itemize}
\end{df*}
\begin{exmpl*}
 Игральный кубик. Честная монетка.
\end{exmpl*}

\begin{df}
 \textit{Случайное событие}  --- это произвольное подмножество $A \subset \Omega$. \textit{Вероятностью} случайного события $A$ называется число

 \begin{align*}
  P(A) = \frac{\# A}{\# \Omega}
 ,\end{align*} где $\# B$ --- количество элементов в конечном множестве $B$.
\end{df}

\begin{prop*}
 Свойства вероятности:
 \begin{enumerate}
  \item $P(\varnothing) = 0$; $P(\Omega) = 1$; $0 \leqslant P(A) \leqslant 1$.
  \item Если $A \subset B$, то $P(A) \leqslant P(B)$.
  \item Если $A \cap B = \varnothing$ (события \textit{не совместны}), то
   \begin{align}
    \label{equation:finite_additivity_of_classic_probability}
    P(A \sqcup B) = P(A) + P(B)
   .\end{align} 
  \item $P(A \cup B) = P(A) + P(B) - P(A \cap B)$.
 \end{enumerate}
\end{prop*}
Оказывается, что из \eqref{equation:finite_additivity_of_classic_probability} следует всё остальное. Действительно, для пункта 2:
\begin{align*}
 P(B) = P(B \setminus A) + P(A) \geqslant 0 + P(A),
\end{align*}  а для пункта 3:

\begin{crly*}
 $P(A \cup B) \leqslant P(A) + P(B)$.
\end{crly*}

\begin{df*}
 $\overline A = \Omega \setminus A$.
\end{df*}
\begin{prop*}
 $\overline A = 1 - P(A)$.
\end{prop*}

\begin{prop}[формула включений-исключений]
 \label{proposition:inclusion_exclusion_formula}
 \begin{align*}
  P(A_1 \cup A_2 \cup \ldots \cup A_m) &= \sum_{i=1}^{m} P(A_i) - \sum_{i < j} P(A_i \cap A_j) + \sum_{i < j < k}  P(A_i \cap A_j \cap A_k) + \ldots \\
  &\ldots + (-1)^{m - 1} P(A_1 \cap \ldots \cap A_m)
 .\end{align*} 
\end{prop}
\begin{proof}
 По индукции. База ($m=1$) очевидно. Переход $m \to m + 1$. Обозначим $B = A_1 \cup \ldots \cup A_m$. Тогда
 \begin{align*}
  P(B \cup A_{m+1}) = \underbrace{P(B)}_{\text{по предположению инд.}} + P(A_{m+1}) - P(B \cap A_{m+1})
 .\end{align*} $P(B)$ раскроем по предположению индукции. Обозначим $B_k = A_k \cap A_{m+1}$. Тогда  $B \cap A_{m+1} = B_1 \cup \ldots B_k$ . Снова по предположению индукции. Получаем:
 \begin{align*}
  &= \sum_{i=1}^{m} P(A_i) - \sum_{i < j \leqslant m} P(A_i \cap A_k) + \ldots + P(A_{m+1}) -\\&- ( \sum_{k=1}^{m} P(A_k \cap A_{m+1}) - \\
  - \sum_{i < j \leqslant m} P(A_i \cap A_j \cap A_{m+1}) + \ldots )
 \end{align*} Получаем ровно то, что нужно.
\end{proof}


\begin{df}
 \textit{Условная вероятность:} если $A \neq \varnothing$, то
 \begin{align*}
  P(B \mid A) = \frac{\# (A \cap B)}{\# A}
 \end{align*} --- \textit{вероятность события $B$ при условии события $A$}. Можно её записать так:
 \begin{align*}
  P(B \mid A) = \frac{P(A \cap B)}{P(A)}
 \end{align*} при условии $P(A) > 0$.
\end{df}
\begin{prop*}\
 \begin{enumerate}
  \item Если $B \supset A$, то $P(B \mid A) = 1$.
  \item $P(\varnothing \mid A) = 0$.
  \item Если $B_1 \cap B_2 = 0$, то $P(B_1 \sqcup B_2 \mid A) = P(B_1 \mid A) + P(B_2 \mid A) $.

   Рассмотреть события $A \cap B_1$ и $A \cap B_2$ и применить аддитивность.
  \item $P(B \mid A) + P(\overline B \mid A) = 1$.
 \end{enumerate} 
 \begin{remrk*}
  $P(B \mid A) + P(B \mid \overline A) \neq 1$. Например: $B = \Omega$, $A$ --- любое непустое. Тогда левая часть равна $2$.
 \end{remrk*}
\end{prop*}

\begin{prop}[формула полной вероятности]
 Пусть $\Omega = \bigsqcup_{k=1}^{m} A_k$, причём $P(A_k) > 0$. Тогда
 \begin{align}
  \label{equation:total_probability_formula}
  P(B) = \sum_{k=1}^{m} P(B \mid A_k) \cdot P(A_k)
 .\end{align} 
\end{prop}
\begin{proof} По конечной аддитивности:
 \begin{align*}
  P(B) = \sum_{k=1}^{m} P(B \cap A_k) = \sum_{k=1}^{m} \frac{P(B \cap A_k) \cdot P(A_k)}{P(A_k)} = \sum_{k=1}^{m} P(B \mid A_k) \cdot P(A_k).
 \end{align*} 
\end{proof}
\begin{exmpl*}
 Есть две урны. В первой урне есть $3$ белых и $5$ чёрных шаров, а во второй --- $5$ белых и $5$ чёрных. Из первой урны мы не глядя достаём два шара и перекладываем во вторую. Затем достаём один шар из второй урны. Какова вероятность того, что он белый?

 Если сейчас заняться формальным определением, то всё будет очень неприятно. Но можно просто применить формулу \eqref{equation:total_probability_formula} полной вероятности.

 Пусть $B$ --- событие <<вынули белый шар>>. Пусть $A_i$ --- событие <<$i$ белых шаров переложили из первой урны во вторую>>. Соответственно, $\Omega = A_0 \sqcup A_1 \sqcup A_2$. Тогда:
 \begin{align*}
  P(B) &= P(B \mid A_0) P(A_0) + P(B \mid A_1) P(A_1) + P(B \mid A_2) P(A_2).
 \end{align*} При этом 
 \begin{align*}
  P(A_0) &= \frac{\binom 5 2}{\binom 8 2} = \frac{5 \cdot 4}{8 \cdot 7} = \frac{5}{14}, \\
  P(A_1) &= \frac{3 \cdot 5}{\binom 8 2} = \frac{15}{28}, \\
  P(A_2) &= \frac{\binom 3 2}{\binom 8 2} = \frac{3 \cdot 2}{8 \cdot 7} = \frac{3}{28}.
 \end{align*} Далее,
 \begin{align*}
  P(B \mid A_0) = \frac{5}{12}, & &P(B \mid A_1) = \frac{6}{12} = \frac{1}{2}, & &P(B \mid A_2) = \frac{7}{12}
 .\end{align*} Ответ:
 \begin{align*}
  P(B)  = \frac{5}{12} \cdot \frac{5}{14} + \frac{1}{2} \cdot \frac{15}{28} + \frac{7}{12} \cdot \frac{3}{28} = \frac{23}{48}.
 \end{align*} 

\end{exmpl*}

\begin{prop}[формула Байеса]
 Пусть $P(A) > 0$ и  $P(B) > 0$. Тогда
 \begin{align}
  \label{equation:bayes_formula}
  P(A \mid B) = \frac{P(B \mid A)P(A)}{P(B)}.
 \end{align} 
\end{prop}
\begin{proof} Подставить $ P(B \mid A) = \frac{P(A \cap B)}{P(A)} $ в правую часть.
\end{proof}
\begin{exmpl*}
 Одно из первых приложений формулы Байеса --- медицинский тест на болезнь. Пусть $A$ --- это <<человек болен>>, а $B$ --- это <<тест положителен>>. При этом могут быть как и false positive, так и false negative. Оказывается, что если болезнь весьма редкая, а тестов много, то почти все тесты будут ложно положительными (это забыли во время коронавируса).
\end{exmpl*}

Скрестим формулы \eqref{equation:total_probability_formula} и \eqref{equation:bayes_formula}.

\begin{thm}[Байеса]
 Пусть $\Omega = \bigsqcup_{k=1}^{m} A_k$, $P(A_k) > 0$ и $P(B) > 0$. Тогда
 \begin{align*}
  P(A_k \mid B) = \frac{P(B \mid A_k)P(A_k)}{\sum_{j=1}^{m} P(B \mid A_j)P(A_j)}
 .\end{align*} 
\end{thm}

\begin{exmpl*}
 Пусть есть две монеты: одна из них симметричная, а вторая с вероятностью $1 / 3$ выдаёт решку, а с $2 / 3$ --- орёл. Мы взяли случайную монету, подбросили, и выпал орёл. Найти вероятность того, что монета симметричная.

 Пусть $A$ --- <<монета симметричная>>, $B$ --- выпал орёл. Тогда
 \begin{align*}
  P(A \mid B) = \frac{P(B\mid A)P(A)}{P(B \mid A)P(A) + P(B \mid \overline A) P(\overline A)}
 .\end{align*} $P(A) = P(\overline A) = 1 / 2$.  $P(B \mid A) = 1 / 2$.  $P(B \mid \overline A) = 2 / 3$. Тогда
 \begin{align*}
  P(A \mid B) = \frac{1 / 2}{1 / 2 + 2 / 3} = \frac{1 / 2}{7 / 6} = \frac{3}{7}
 .\end{align*} 

\end{exmpl*}

Независимые события. Событие $B$ \textit{не зависит} от $A$, если $P(B) = P(B \mid A)$. Но
\begin{align*}
 P(B \mid A) = \frac{P(A\cap B)}{P(A)}
.\end{align*} Поэтому удобнее определять через произведение вероятностей.

\begin{df}
 $A$ и $B$ --- \textit{независимые события}, если $P(A \cap B) = P(A)P(B)$.
\end{df}
\begin{df}
 События $A_1, A_2, \ldots, A_m$ \textit{независимы в совокупности}, если
 \begin{align*}
  P(A_{i_1} \cap A_{i_2} \cap \ldots \cap A_{i_k}) = P(A_{i_1}) P(A_{i_2}) \ldots P(A_{i_k})
 \end{align*} для любых наборов различных индексов $i_1, i_2, \ldots, i_k$.
\end{df}
Соглашение: по умолчанию \textit{независимость} означает <<независимость в совокупности>>.
\begin{remrk*}
 Независимость в совокупности --- это не то же самое, что и попарная независимость (хотя из первого следует второе).
\end{remrk*}
\begin{exmpl*}
 Подбросили дважды игральный кубик. Пусть $A$ --- <<при первом броске выпал чёт>>, $B$ --- <<при втором броске выпал чёт>>. Пусть $C$ --- <<выпала чётная сумма>>. $\Omega = \left\{ (i,j) \mid 1 \leqslant i,j \leqslant 6 \right\}$.
 \begin{align*}
  P(A) = P(B) = P(C) = \frac{1}{2} \\
  P(A \cap B) = \frac{1}{4} = P(A) P(B) \\
  P(A \cap C) = P(B \cap C) = P(A \cap B) = \frac{1}{4}
 .\end{align*} Попарная независимость есть. Но
 \begin{align*}
  P(A \cap B \cap C) = \frac{1}{4} \neq P(A)P(B)P(C)
 .\end{align*} Независимости в совокупности нет!
\end{exmpl*}

\begin{thm}[Эрдёша-Мозера]
 Есть одно-круговой турнир, где играют $n$ волейбольных команд. Каждая сыграла с каждой. Пусть $k$ --- наибольшее число, для которого заведомо можно найти такие команды  $A_1, \ldots, A_k$, что $A_i$ выиграла у $A_j$ если $i < j$.

 Тогда $k \leqslant 1 + \left\lfloor 2 \log_2 n  \right\rfloor$.
\end{thm}
\begin{proof}
 Вероятностный метод. Рассмотрим случайный турнир (всего турниров $2^{\binom n 2}$).
 \begin{align*}
  P(A \text{ выиграла у } B) = 1 / 2
 .\end{align*} 
 \begin{align*}
  P(\text{команды } A_1, \ldots, A_k \text{ подходят}) &= \frac{1}{2^{\binom k 2}} \\
  P(\text{какие-то $k$ команд подходят}) &\leqslant \sum_{\text{набор из $k$ команд}} P(A_1, \ldots, A_k \text{ подходят }) = \\
  &= \binom n k \cdot k!  \cdot \frac{1}{2^{k(k-1) / 2}}
 .\end{align*} Предположим, что $k \geqslant 2 + \left\lfloor 2 \log_2 n \right\rfloor $. Тогда
 \begin{align*}
  \leqslant \frac{n (n-1) \ldots (n-k+1)}{2^{k(k-1) / 2}} < \frac{n^{k}}{\left(2^{\frac{k-1}{2}}\right)^{k}} = (\frac{n}{2^{\frac{k-1}{2}}})^{k} < 1
 .\end{align*} Тогда
 \begin{align*}
  P(\text{никакие $k$ команд не подходят}) > 0
 .\end{align*} Следовательно, существует турнир, в котором никакие $k$ команд не подходят.
\end{proof}

\begin{df*}
 
Небольшое обобщение модели: пусть $\Omega = \left\{ \omega_1, \ldots, \omega_n \right\}$, и $P(\omega_j) = p_j \in (0,1]$, причём $p_1 + \ldots + p_n = 1$. Тогда определим
\begin{align*}
 P(A) = \sum_{\omega_j \in A} p_j
.\end{align*}
\end{df*}

Чтобы все свойства были верными, нужно проверить лишь конечную аддитивность:
\begin{align*}
 P(A \sqcup B) = P(A) + P(B)
,\end{align*} что очевидно.
