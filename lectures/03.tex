% 2023.02.28: lecture 03
\documentclass[../main.tex]{subfiles}
\begin{document}

\begin{exmpl}[задача о театре]
 Есть театр, в который имеется два входа. В театре $n=1600$ мест. У каждого входа есть гардероб. Зрители заходят через один из входов, и идут оставлять вещи в соответствующий гардероб. Мы хотим сделать так, чтобы не чаще, чем раз в месяц переполнялся хотя бы один из гардеробов. 

 Есть схема Бернулли, $p = \frac{1}{2}$. Пусть $C \geqslant 0$ --- это сколько дополнительных мест (поверх необходимых $800$) мы предоставили в каждый из гардеробов. Тогда оба гардероба не переполнятся, если и только если $800 - C \leqslant S_n \leqslant 800 + C$. Значит, нам нужно выполнить условие
 \begin{align*}
  P(800 - C \leqslant S_n \leqslant 800 + C) \geqslant \frac{29}{30}
 \end{align*} и минимизировать при этом $C$. По интегральной теореме \ref{theorem:intergram_theorem_muavr_laplas} Муавра-Лапласа вероятность равна
 \begin{align*}
  P(800 - C \leqslant S_n \leqslant 800 + C) &= P(-C \leqslant S_n - np \leqslant C) = \\
  &= P\left(-\frac{C}{\sqrt{npq}} \leqslant \frac{S_n - np}{\sqrt{npq}} \leqslant \frac{C}{\sqrt{npq}}\right) = \\
  &= P \left( -\frac{C}{20} \leqslant \frac{S_n - np}{\sqrt{npq}} \leqslant \frac{C}{20} \right) \approx \\
  &\approx \frac{1}{\sqrt{2\pi}} \int\limits_{-C / 20}^{C / 20}  e^{-t^{2} / 2} \, dt.
 \end{align*} Значит нам нужно выполнить условие
 \begin{align*}
  \Phi\left(\frac{C}{20}\right) - \Phi(0) = \frac{1}{\sqrt{2\pi}}\int\limits_{0}^{C / 20} e^{-t^{2} / 2} \, dt \geqslant \frac{29}{60}
 .\end{align*} Табличка говорит, что нужно взять $\frac{C}{20} \approx 2.13$. Тогда
 \begin{align*}
  C = 20 \cdot 2.13 \implies C = 43 
 \end{align*} мест достаточно.
\end{exmpl}

\begin{exmpl}[случайное блуждание по целым точкам]
 Частица перемещается по целым точкам на прямой. С вероятностью $p$ она идёт в положительную сторону, а с вероятностью $q = 1 - p$ --- в отрицательную. Стартуем в нуле. Нас интересует, как будет ходить частица.

 Пусть $a_n$ --- позиция частицы на шаге $n$ ($a_0 = 0$). Тогда
 \begin{align*}
  a_{n+1} = \begin{cases}
   a_n + 1, \text{ с вероятностью } p,  \\
   a_n - 1, \text{ с вероятностью } q.
  \end{cases}
 \end{align*} Здесь удобнее понимать это так: с вероятностью $p$ мы прибавляем $2$, но всегда вычитаем $1$. Тогда
 \begin{align*}
  a_n = 2S_n - n.
 \end{align*} Таким образом, вероятность оказаться в точке $k$ через $n$ шагов равна
 \begin{align*}
  P(a_n = k) = \begin{cases}
   0, \text{ если } k \not \equiv n \pmod 2;  \\
   P(S_n = \frac{n+k}{2}) = \binom {\frac{n+k}{2}} n p^{\frac{n+k}{2}} q^{\frac{n-k}{2}}, \text{ иначе. }
  \end{cases} 
 \end{align*} 
\end{exmpl}

\begin{exmpl*}[числа Рамсея]
 Число Рамсея $R(n,k)$  --- это такое наименьшее натуральное число $m$, что в любом графе на $m$ вершинах либо есть полный подграф на $n$ вершинах ($n$-клика), либо пустой подграф на $k$ вершинах ($k$-независимое множество).
\end{exmpl*}
Следующая теорема --- ещё одна иллюстрация вероятностного метода.
\begin{thm}[Эрдёша]
 Если $\binom m k \cdot 2^{1 - \frac{k(k-1)}{2}} < 1$, то $R(k,k) > m$. В частности, $R(k,k) > 2^{k / 2}$.
\end{thm}
\begin{proof}[\normalfont\textsc{Доказательство}]
 Рассмотрим случайный граф на $m$  вершинах: для каждой из $\frac{m(m-1)}{2}$ пар вершин возьмём соответствующее ребро с вероятностью $\frac{1}{2}$ (независимо от других пар). Тогда
 \begin{align*}
  P(\text{вершины }a_1, \ldots, a_k\text{ подходят}) =  \frac{2^{\binom m 2 - \binom k 2}}{2^{\binom m 2}} + \frac{2^{\binom m 2 - \binom k 2}}{2^{\binom m 2}} =  2^{1-\frac{k(k-1)}{2}}.
 \end{align*} Теперь, пользуясь неравенством <<union-bound>>, оценим
 \begin{align*}
  P(k \text{ вершин подходят}) = P \left( \bigcup_{a_1, \ldots, a_k} a_1, \ldots, a_k \text{ подходят} \right) \leqslant \\
  \leqslant \sum_{a_1, \ldots, a_k}  P(a_1, \ldots, a_k \text{ подходят}) = \binom m k \cdot 2^{1 - \frac{k(k-1)}{2}} < 1.
 \end{align*} Иными словами,
 \begin{align*}
  P(\text{никакие $k$ вершин не подходят}) > 0.
 \end{align*} Следовательно, существует граф, в котором никакие  $k$  вершин не подходят.

 Теперь разберёмся с <<в частности>>. Подставим $m \leqslant 2^{k / 2}$ и проверим неравенство:
 \begin{align*}
  \binom m k \cdot 2^{1 - \frac{k(k-1)}{2}} &= \frac{m (m-1) \ldots (m - k + 1)}{k!}2^{1 - \frac{k(k-1)}{2}} \leqslant \frac{m^{k}}{k!} 2^{1-\frac{k(k-1)}{2}} \leqslant \\
  &\leqslant \frac{2^{\frac{k^{2}}{2}}}{k!}2^{1 - \frac{k^{2}}{2} + \frac{k}{2}} = \frac{2^{\frac{k}{2} + 1}}{k!}. \\
 \end{align*} Число $\frac{2^{\frac{k}{2} + 1}}{k!} < 1$ при $k \geqslant 3$. Докажем по индукции. База ($k = 3$) проверяется подстановкой; при переходе от $k$ к $k + 1$ числитель домножается  на $\sqrt 2$, а знаменатель --- на $(k + 1) > \sqrt 2$.
\end{proof}

\newpage
\part{Общая теория вероятностей.}
\section{Колмогоровская модель теории вероятностей.}

Сейчас мы построим <<более взрослую>> теорию вероятностей, вовсю опирающуюся на теорию меры, которую мы усердно изучали в третьем семестре матана.

\begin{df}[вероятностное пространство]
 \textit{Вероятностным пространством} называется тройка $(\Omega, \mathcal F, P)$, где
 \begin{itemize}
  \item $\Omega$ --- произвольное множество, называемое \textit{пространством элементарных исходов}.
  \item $\F \subset 2^{\Omega}$ --- $\sigma$-алгебра подмножеств $\Omega$, называемая \textit{пространством событий}. Элементы $\F$ называются \textit{случайными событиями} (или просто \textit{событиями}).
  \item $P \colon\, \F \to [0,1]$ --- \textit{вероятностная мера} на $\F$. То есть функция $P$ счётно-аддитивна:
   \begin{align*}
    P \left( \bigsqcup_{k=1}^{\infty}A_k \right) = \sum_{k=1}^{\infty}P(A_k), \qquad A_k \in \mathcal F,
   \end{align*} неотрицательна, и $P(\Omega) = 1$.
 \end{itemize}
\end{df}

По сути вероятность --- это конечная мера (а вероятностное пространство --- пространство с конечной мерой), условие $ P(\Omega) = 1 $ можно воспринимать лишь как нормировку, сделанную для удобства.

Как и не все множества измеримы в теории меры, так и не всякое подмножество $ \Omega $ имеет вероятность (то есть является \textit{событием}). Тем не менее, измеримых множеств достаточно много.

Покажем явно, что классическая модель теории вероятностей из предыдущей главы является частным случаем нового определения.

\begin{exmpl}[дискретное вероятностное пространство]
 Пусть пространство элементарных исходов $\Omega = \left\{ \omega_1, \omega_2, \ldots \right\}$ не более, чем счётно. Пусть $p_1, p_2, \ldots \in [0,1]$ --- вероятности элементарных исходов $\omega_1, \omega_2, \ldots$, причём $p_1 + p_2 + \ldots  = 1$. Тогда положим $\F = 2^{\Omega}$ и для всякого $A \subset \Omega$
 \begin{align*}
  P(A) = \sum_{\omega_i \in A} p_i.
 \end{align*} Полученное вероятностное пространство называется \textit{дискретным}.
\end{exmpl}

Зафиксируем теперь вероятностное пространство $(\Omega, \F, P)$ и введём все те же самые определения, какие мы вводили для классической модели. 

\begin{df}[условная вероятность]
 Пусть $A \in \F$ --- событие, такое, что $P(A) > 0$. Тогда \textit{вероятностью события $B \in \mathcal F$ при условии события $A$} называется число
 \begin{align*}
  P(B \mid A) = \frac{P(A\cap B)}{P(A)}.
 \end{align*}
\end{df}

\begin{df}[независимые события]
 Говорят, что два события $A, B \in \F$ \textit{независимы}, если $P(A \cap B) = P(A) P(B)$.
\end{df}
\begin{df}[независимость в совокупности]\

 Говорят, что события $A_1, \ldots, A_n \in \F$ \textit{независимы в совокупности}, если для любых различных индексов $i_1, \ldots, i_k \in [n]$ верно
 \begin{align*}
  P(A_{i_1}\cap \ldots \cap A_{i_k}) = P(A_{i_1}) \cdot \ldots \cdot P(A_{i_k}).
 \end{align*} 
\end{df}
\begin{conventn*}
 Как и ранее, когда мы будем говорить, что события $A_1, \ldots, A_n$ \textit{независимы}, по умолчанию мы будем иметь в виду независимость в совокупности. \textit{Попарную независимость} мы будем иметь в виду, только если так будет сказано явно.
\end{conventn*}
\begin{remrk}
 \label{remark:independed_complement}
 События $A_1, \ldots, A_n \in \F$ независимы тогда и только тогда, когда для всех наборов событий $\left\{ B_k \right\}_{k=1}^{n}$, где $B_k = A_k$ или $B_k = \overline{A_k}$, верно равенство
 \begin{align}
  \label{equation:remark:independent_complement}
  P(B_1 \cap \ldots \cap B_n) = P(B_1) \cdot\ldots\cdot P(B_n).
 \end{align}
\end{remrk}
\begin{proof}[\normalfont\textsc{Доказательство}]
 Чтобы доказать в одну сторону (из независимости в \eqref{equation:remark:independent_complement}), достаточно показать, что если события $A_1, \ldots, A_{n-1}, A_n$ независимы, то независимы и события $A_1, \ldots, A_{n-1}, \overline{A_n}$. Действительно, если это верно, то мы можем любое подмножество событий заменить на их дополнения без потери независимости, и получить формулу \eqref{equation:remark:independent_complement} для любого набора $\{B_{k}\}_{k=1}^{n} $. Проверим это: пусть $i_1, \ldots, i_{l} \in [n-1]$ --- различные индексы. Тогда по формуле полной вероятности
 \begin{align*}
  P(A_{i_1} \cap \ldots \cap A_{i_l} \cap A_n) + P(A_{i_1} \cap \ldots \cap A_{i_l} \cap \overline{A_n}) = P(A_{i_1} \cap \ldots \cap A_{i_l}).
 \end{align*} Тогда
 \begin{align*}
  P(A_{i_1} \cap \ldots \cap A_{i_l}\cap \overline{A_n}) &= P(A_{i_1} \cap \ldots \cap A_{i_l}) - P(A_{i_1} \cap \ldots \cap A_{i_l} \cap A_n) = \\
  &= P(A_{i_1})  \ldots  P(A_{i_l}) - P(A_{i_1})  \ldots  P(A_{i_l})  P(A_n) = \\
  &= P(A_{i_1})  \ldots  P(A_{i_l})  (1 - P(A_n)) \\
  &= P(A_{i_1})  \ldots  P(A_{i_l})  P(\overline{A_n}).
 \end{align*} Независимость проверена.

 Наоборот, пусть есть неравенство \eqref{equation:remark:independent_complement} для любого набора $B_k$. Возьмём любой набор различных индексов $I = \left\{i_1, \ldots, i_l \right\} \subset [n]$. Пусть $[n] \setminus I = \left\{ j_1, \ldots, j_{n-l} \right\}$ --- оставшиеся индексы. Тогда по формуле полной вероятности:
 \begin{align*}
  P(A_{i_1} \cap \ldots \cap A_{i_l}) = \sum_{\{B_{j_t}\}_{t=1}^{n-l} } P(A_{i_1} \cap \ldots \cap A_{i_l} \cap B_{j_1} \cap \ldots \cap B_{j_{n-l}}),
 \end{align*} где сумма берётся по всем наборам $ \{B_{j_t}\}_{t=1}^{n-l} $ таким, что $B_{j_t} = A_{j_t}$ или $B_{j_t} = \overline{A_{j_t}}$. По неравенству \eqref{equation:remark:independent_complement}:
 \begin{align*}
  P(A_{i_1} \cap \ldots \cap A_{i_l}) &= \sum_{\{B_{j_t}\}_{t=1}^{n-l} } P(A_{i_1}) \ldots P(A_{i_l}) \cdot P(B_{j_1}) \ldots P(B_{j_t}) = \\
  &= P(A_{i_1}) \ldots P(A_{i_l}) \cdot \sum_{\{B_{j_t}\}_{t=1}^{n-l} } P(B_{j_1}) \ldots P(B_{j_t}) = \\
  &= P(A_{i_1}) \ldots P(A_{i_l}) \cdot \prod_{t=1}^{n-l} (P(A_{j_t}) + P(\overline{A_{j_t}})) = \\
  &= P(A_{i_1}) \ldots P(A_{i_l}).
 \end{align*} 
\end{proof}

Таким образом, среди независимых событий можно свободно переходить к их дополнениям.

\begin{df}[независимость бесконечного числа событий]
 Говорят, что события $A_1, A_2, \ldots \in \F$ \textit{независимы}, если любое их конечное подмножество независимо. Иными словами, для любого конечного набора различных индексов $i_1, \ldots, i_n$ верно
 \begin{align*}
  P(A_{i_1} \cap \ldots \cap A_{i_n}) = P(A_{i_1}) \cdot \ldots \cdot P(A_{i_n}).
 \end{align*} 
\end{df}
\begin{remrk}
 \label{remark:independent_infinite_product}
 Если события $A_1, A_2, \ldots \in \F$  независимы, то
 \begin{align*}
  P \left( \bigcap_{k=1}^{\infty} A_k \right) = \prod_{k=1}^{\infty} P(A_k).
 \end{align*} 
\end{remrk}
\begin{proof}
 Для любого конечного $n$ имеем
 \begin{align*}
  P \left( \bigcap_{k=1}^{n} A_k \right) = \prod_{k=1}^{n} P(A_k)
 .\end{align*} В пределе при $n \to \infty$ правая часть стремится к бесконечному произведению. Так как множества $ \bigcap_{k=1}^{n}A_k  $ убывают при $ n \to \infty $, то по непрерывности меры снизу левая часть стремится к  $ P( \bigcap_{k=1}^{\infty} A_k)  $.
\end{proof}
\begin{lm}[%
 Бореля-Кантелли]
 \label{lemma:borel_cantelli}
 Пусть $A_1, A_2, \ldots \in \F$  --- случайные события. Обозначим за $ B \in \F$ событие <<среди событий $ A_k $ случилось бесконечно много событий>>:
 \begin{align*}
  B = \bigcap_{n=1}^{\infty} \bigcup_{k \geqslant n} A_k.
 \end{align*} Также обозначим неотрицательный ряд вероятностей
 \begin{align*}
  S = \sum_{k=1}^{\infty} P(A_k).
 \end{align*}  Тогда:
 \begin{enumerate}
  \item \label{enum1:lemma:borel_cantelli} Если ряд $ S $ сходится, то $P(B) = 0$.
  \item \label{enum2:lemma:borel_cantelli} Если события $A_k$  независимы и ряд $ S $ расходится к $ +\infty $, то $P(B) = 1$.
 \end{enumerate} 
\end{lm}
\begin{proof}[\normalfont\textsc{Доказательство}]\
 \begin{enumerate}
  \item Этот пункт был доказан на матане, но доказательство простое:
   \begin{align*}
    P(B) \leqslant P \left( \bigcup_{k \geqslant n} A_k \right) \leqslant \sum_{k \geqslant n} P(A_k) \to 0,
   \end{align*} где правая часть стремится к нулю как хвост сходящегося ряда $ S $.

  \item Докажем второй пункт. Обозначим $B_k = \overline {A_k}$. События $B_1, B_2, \ldots$ независимы по замечанию \ref{remark:independed_complement}. По замечанию \ref{remark:independent_infinite_product} для любого $ n \geqslant 1$:
   \begin{align*}
    P \left( \bigcap_{k \geqslant n} B_k \right) = \prod_{k \geqslant n} P(B_k) = \prod_{k \geqslant n} (1 - P(A_k)).
   \end{align*} Пользуясь неравенством $ 1 + x \leqslant e^{x} $ получаем 
   \begin{align*}
    P \left( \bigcap_{k \geqslant n} B_k \right) \leqslant \prod_{k \geqslant n} e^{-P(A_k)} = \exp \left(-\sum_{k \geqslant n} P(A_k)\right) = 0,
   \end{align*} так как $ \sum_{k \geqslant n} P(A_k) = +\infty $ для любого $ n $ (хвост расходящегося неотрицательного ряда). Следовательно,
   \begin{align*}
    P \left( \overline { \bigcap_{k \geqslant n} B_k } \right) = 1 \implies P \left( \bigcup_{k \geqslant n} A_k \right) = 1
   \end{align*} для всякого $ n \geqslant 1 $. Кроме того, множества $\bigcup_{k \geqslant n} A_k$ убывают при $ n \to \infty $, поэтому по непрерывности меры снизу имеем
   \begin{align*}
    P(B) = P \left( \bigcap_{n=1}^{\infty} \bigcup_{k \geqslant n} A_k \right) = \lim_{n \to \infty} P \left( \bigcup_{k \geqslant n}^{\infty} A_k \right) = 1.
   \end{align*}
 \end{enumerate}
\end{proof}
\begin{crly*}[закон нулей и единиц]
 Пусть события $A_1, A_2, \ldots \in \F$ независимы. Тогда вероятность того, что случилось бесконечное число из них равна либо нулю, либо единице. 
\end{crly*}

\section{Случайные величины}

Как и не любое подмножество $ \Omega $ является случайным событием, так и не любая функция из $ \Omega $ будет случайной величиной. Дадим формальное определение.

\begin{df}[%
 случайная величина]
 Пусть $(\Omega, \F, P)$ --- вероятностное пространство. Всякая измеримая функция $\xi \colon\, \Omega \to \R$ называется \textit{случайной величиной}.
\end{df}

Напомним, что функция $ \xi \colon\, \Omega \to \R $ называется \textit{измеримой}, если $ \xi^{-1}((a, +\infty)) \in \mathcal F $ для всякого $ a \in \R \cup \{\pm \infty\} $. Как мы увидели на теории меры, почти все  встречающиеся на практике функции так или иначе измеримы, то есть это условие довольно-таки слабое.

\begin{df}[%
 распределение случайной величины]
 \textit{Распределением случайной величины} $\xi$ называется мера $ P_{\xi} \colon\, \mathcal B_1 \to [0, 1] $ на борелевских подмножествах $\R$, заданная формулой
 \begin{align*}
  P_{\xi}(A) = P(\xi \in A) = P(\xi^{-1}(A)).
 \end{align*} 
\end{df}
\begin{remrk*}
 Легко видеть, что распределение всякой случайной величины является мерой: ведь под действием прообраза $ \xi^{-1} $ всякое счётное дизъюнктное объединение $ A_1 \sqcup A_2 \sqcup \ldots $ измеримых множеств переходит в счётное дизъюнктное объединение $ \xi^{-1}(A_1) \sqcup \xi^{-1}(A_2) \sqcup \ldots $ измеримых множеств. Более того, распределение является вероятностной мерой, так как $ \xi^{-1}(\R) = 1 $.
\end{remrk*}
\begin{df*}
 Говорят, что случайные величины $\xi$, $\eta$ \textit{одинаково распределены}, если их распределения совпадают (как меры): $P_{\xi} = P_{\eta}$. При этом мы допускаем, что величины могут быть заданы на разных вероятностных пространствах.
\end{df*}
\begin{remrk}
 \label{remark:_distribution_eq}
 Из единственности стандартного продолжения меры (теорема Каратеодори), равенство распределений $P_{\xi} = P_{\eta}$  достаточно проверять на ячейках:
 \begin{align*}
  P_{\xi} \left(a, b\right] = P_{\eta} \left(a, b\right]  \iff P(\xi \leqslant b) - P(\xi \leqslant a) =  P(\eta \leqslant b) - P(\eta \leqslant a)
 .\end{align*} Более того, достаточно, чтобы для всех $a \in \R$ выполнялось
 \begin{align*}
  P(\xi \leqslant a) = P(\eta \leqslant a).
 \end{align*} 
\end{remrk}
\begin{df}[%
 функция распределения]
 \textit{Функцией распределения} случайной величины $\xi$ называется функция $F_{\xi} \colon\, \R \to \R$, определённая так:
 \begin{align*}
  F_{\xi}(x) = P(\xi \leqslant x).
 \end{align*} 
\end{df}
\begin{remrk*}
 Как следует из замечания \ref{remark:_distribution_eq}, если случайные величины $ \xi $ и $ \eta $ имеют одинаковые функции распределения, то они одинаково распределены (сами их распределения равны).
\end{remrk*}
\begin{prop*}[cвойства функций распределения]\
 \begin{enumerate}
  \item $0 \leqslant F_{\xi} \leqslant 1$.
  \item \label{enum2:prop:distribution_function} $F_{\xi}$  нестрого возрастает.
  \item \label{enum3:prop:distribution_function} $\lim\limits_{x \to +\infty} F_{\xi}(x) = 1$  и $\lim\limits_{x \to -\infty} F_{\xi}(x) = 0$.
   \begin{proof}[\normalfont\textsc{Доказательство}]
    Непрерывность меры снизу (стремление к $\varnothing$) и сверху (стремление к $\Omega$).
   \end{proof}
  \item \label{enum4:prop:distribution_function} $F_{\xi}$  непрерывна справа.
   \begin{proof}[\normalfont\textsc{Доказательство}]
    Возьмём любую убывающую последовательность точек $y_n \to x$, $y_n > x$. Тогда
    \begin{align*}
     F_{\xi}(y_n) = P(\xi \leqslant y_n) = P(A_n)
    .\end{align*} Множества $A_n$ убывают. По непрерывности меры снизу получаем
    \begin{align*}
     F_{\xi}(y_n) \to P \left( \bigcap_{n=1}^{\infty} A_n \right) = P(\xi \leqslant x).
    \end{align*} 
   \end{proof}
  \item $F_{\xi + c}(x) = F_{\xi}(x - c)$.
  \item Если $c > 0$, то $F_{c\xi}(x) = F_{\xi}(x / c)$.
  \item $P[\xi < x] = \lim\limits_{y \to x-} F_{\xi}(y)$.
 \end{enumerate}
\end{prop*}
\begin{remrk}
 Любая функция, удовлетворяющая свойствам \ref{enum2:prop:distribution_function}, \ref{enum3:prop:distribution_function}, \ref{enum4:prop:distribution_function} есть функция распределения некоторой случайной величины.
\end{remrk}
\begin{proof}[\normalfont\textsc{Доказательство}]
 Пусть $F$ --- такая функция. Рассмотрим функцию множества
 \begin{align*}
  \nu \left(a, b\right] = F(b) - F(a),
 \end{align*} заданную на полукольце ячеек $ \mathcal P_1 $. Функция $ \nu $ является неотрицательной (так как $ F $ нестрого возрастает по пункту \ref{enum2:prop:distribution_function}), а также счётно-аддитивной (так как она непрерывна справа по пункту \ref{enum3:prop:distribution_function}, но это требует доказательства). По теореме Каратеодори существует мера $ \nu $ --- её стандартное продолжение на некоторую $ \sigma $-алгебру $ \mathfrak A $ подмножеств $ \R $. При этом мера $ \nu $ --- вероятностная, так как по пункту \ref{enum4:prop:distribution_function}
 \begin{align*}
  \nu(\R) = \lim_{x \to +\infty} \nu \left(-x, x\right] = \lim_{x \to +\infty} F(x) - \lim_{x \to -\infty} F(x) = 1. 
 \end{align*} Таким образом, мы получили вероятностное пространство $ (\R,\mathfrak A, \nu) $. Случайная величина $ \xi \colon\, \R \to \R $, $ \xi(\omega) = \omega $ как раз даёт нужное распределение $ F $.
\end{proof}

\begin{df}[%
 дискретное распределение]
 Говорят, что случайная величина обладает \textit{дискретным распределением}, если её область значений не более, чем счётная. В таком случае случайная величина называется \textit{дискретной}.
\end{df}
\begin{prop*}
 Пусть $\xi \colon\, \Omega \to \left\{ y_1, y_2, \ldots \right\}$ --- дискретная случайная величина. Тогда для любого борелевского множества $ A \subset \R $
 \begin{align*}
  P_{\xi}(A) = \sum_{y_k \in A}  P(\xi = y_k)
 .\end{align*} Действительно, это следует из счётно-аддитивности функции распределения $ P_{\xi} $. Таким образом, дискретная случайная величина (при фиксированном множестве значений $ \left\{ y_1, y_2, \ldots \right\} $) однозначно определяется счётным набором вероятностей $ p_1, p_2, \ldots \in [0,1] $, $ p_1 + p_2 + \ldots = 1 $, $ p_n = P(\xi = y_n) $.

 В частности, функцию распределения величины $ \xi $ можно выразить так:
 \begin{align*}
  F_{\xi}(x) = \sum_{y_k \leqslant x} P(\xi = y_k).
 \end{align*} 
\end{prop*}

\begin{df}[непрерывное распределение]
 Говорят, что случайная величина $\xi$ \textit{обладает непрерывным распределением}, если $P(\xi = x) = 0$ для всех $x \in \R$.
\end{df}

\begin{remrk}
 Случайная величина $ \xi $ обладает непрерывным распределением тогда и только тогда, когда её функция распределения $ F_{\xi} $ является непрерывной.
\end{remrk}
\begin{proof}[\normalfont\textsc{Доказательство}]
 Пусть $ \xi $ обладает непрерывным распределением. Непрерывность справа есть и так, поэтому достаточно проверить непрерывность слева в произвольной точке $ x \in \R $:
 \begin{align}
  \label{equation:remark:left_continous_of_distribution}
  \lim_{y \to x-} F_{\xi}(y) &= P(\xi < x) = P(\xi \leqslant x) - P(\xi = x) = P(\xi \leqslant x) = F_{\xi}(x).
 \end{align} Наоборот, если функция распределения непрерывна, то верна цепочка \eqref{equation:remark:left_continous_of_distribution} для любой точки $ x \in \R $, из чего следует $ P(\xi = x) = 0 $.
\end{proof}

Непрерывности распределения порой бывает недостаточно: удобнее работать с более частным случаем.

\begin{df}
 Говорят, что  случайная величина $\xi$ \textit{абсолютно непрерывна} (\textit{обладает абслютно непрерывным распределением}), если существует неотрицательная измеримая функция $p_{\xi} \colon\, \R \to [0, +\infty)$, называемая \textit{плотностью распределения} случайной величины $ \xi $,  такая, что для любой точки $ x \in \R $
 \begin{align}
  \label{equation:density_def}
  F_{\xi}(x) = \int\limits_{-\infty}^{x} p_{\xi}(t)\,dt
 .\end{align}
\end{df}
\begin{prop}
 Пусть случайная величина $ \xi $ обладает абсолютно непрерывным распределением; $ p_{\xi} $ --- её плотность распределения. Тогда
 \begin{enumerate}
  \item Для любого борелевского множества $ A \subset \R $ выполнено
   \begin{align}
    \label{equation:value_of_absolute_continous_distribution}
    P(\xi \in A) = \int\limits_{A} p_{\xi} (t) \, dt.
   \end{align} 
  \item $ \int_{\R} p_{\xi}(t)\,dt = 1   $.
  \item При почти всех точках $ t \in \R $ выполнено $ F'_{\xi}(t) = p_{\xi}(t) $.
 \end{enumerate}
\end{prop}
\begin{proof}[\normalfont\textsc{Доказательство}]\
 \begin{enumerate}
  \item Равенство \eqref{equation:value_of_absolute_continous_distribution} можно рассматривать как равенство двух мер: $ P_\xi(A) = \int_{A} p_\xi \, d\lambda_1  $. По теореме Каратеодори \eqref{equation:value_of_absolute_continous_distribution} достаточно проверять на ячейках, а это немедленно следует из \eqref{equation:density_def}.
  \item Подставить $ A = \R $ в \eqref{equation:value_of_absolute_continous_distribution}.
  \item Теорема о точках Лебега.
 \end{enumerate}
\end{proof}

\end{document}
