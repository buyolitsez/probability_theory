% 2023.03.07: lecture 04

Рассмотрим одну полезную формулу из теории меры: интеграл по мере, имеющей плотность.

\begin{thm}[%
интеграл по мере, имеющей плотность]
\label{theorem:integral_on_measure_with_density}
Пусть $ w \colon X \to [0,\infty] $ --- измеримая, неотрицательная функция, заданная на пространстве с мерой $ (X,\mathfrak A,\mu) $. Пусть
 \begin{align*}
  \nu(A) = \int\limits_{A} w\,d\mu 
 \end{align*} --- мера (функцию $ w $ называют плотностью меры $ \nu $). Тогда для любой функции $ f $ верно
 \begin{align}
  \label{equation:integral_on_measure_with_density}
  \int\limits_{X} f\,d\nu = \int\limits_{X} f w \, d\mu,
 \end{align} если $ f \geqslant 0 $ или если $ f\omega $ суммируема относительно $ \mu $.
\end{thm}
\begin{conventn*}
 Пусть $ A \subset X $ --- измеримое подмножество. В теории вероятностей функцию
 \begin{align*}
  f \colon\, x \mapsto \begin{cases}
   1, \text{ если } x \in A; \\
   0, \text{ если } x \notin A,
  \end{cases} 
 \end{align*} называют \textit{индикаторной функцией множества $ A $} (а не характеристической, как мы её называли на матане) и обозначают $ \Ind_A = f $.
\end{conventn*}
\begin{proof}[\normalfont\textsc{Доказательство}]\
 \begin{itemize}
  \item Пусть $ f = \Ind_A $ --- индикаторная функция. Тогда
   \begin{align*}
    \int\limits_{X} \Ind_A \, d\nu = \nu(A) = \int\limits_{A} w\,d\mu = \int\limits_{X} \Ind_A w\,d\mu.    
   \end{align*} 
  \item По линейности \eqref{equation:integral_on_measure_with_density} верно для простых неотрицательных функций $ f $.
  \item Пусть теперь $ f $ неотрицательна и измерима. По теореме об аппроксимации можно взять последовательных простых неотрицательных функций $\{f_{n}\}_{n=1}^{\infty} $, которые поточечно возрастают к $ f $. По предыдущему пункту для каждого $ n \geqslant 1 $ имеем
   \begin{align*}
    \int\limits_{X} f_n \,d\nu = \int\limits_{X} f_n w \, d\mu.  
   \end{align*} По теореме Леви левая часть стремится к $ \int_{X} f\,d\nu  $, а правая --- к $ \int_{X} fw\,d\mu $ (ведь функции $ f_n w $ также измеримы, неотрицательны и поточечно возрастают к функции $ fw $). Равенство \eqref{equation:integral_on_measure_with_density} доказано для неотрицательных измеримых функций.

  \item Для суммируемых функций произвольного знака нужно рассмотреть функции $ f_\pm = \max(\pm f, 0) $, воспользоваться предыдущим пунктом и линейностью.
 \end{itemize}
\end{proof}

Рассмотрим несколько канонических примеров вероятностных распределений. Начнём с дискретных распределений.

\begin{exmpl}[биномиальное распределение] Пусть есть число $ p \in [0,1] $ и целое число $ n \geqslant 0 $. \textit{Биномиальным распределением} называется распределение случайной величины $ \xi \colon\, \Omega \to \left\{ 0, 1, \ldots, n \right\} $, для которой
 \begin{align*}
  P(\xi = k) = \binom n k p^{k}(1-p)^{k}
 \end{align*} при всех $ 0 \leqslant k \leqslant n $. Обозначение: $ \xi \sim \Binom(n,p) $.

 Биномиальным распределением обладает величина $ S_n $ --- количество успехов с схеме Бернулли с $ n $ испытаниями и вероятностью успеха $ p $.
\end{exmpl}
\begin{exmpl}[распределение Пуассона]
 Пусть есть число $ \lambda > 0 $. \textit{Распределением Пуассона} называется распределение случайной величины $ \xi\colon\,\Omega\to \left\{ 0,1,2,\ldots \right\} $, для которой
 \begin{align*}
  P(\xi = k) = \frac{\lambda^{k}}{k!} e^{-\lambda}.
 \end{align*} Обозначение: $ \xi \sim  \Poisson(\lambda)$.

 В теореме \ref{theorem:poisson} Пуассона было показано, что при условии
 \begin{align*}
  \lim_{n \to \infty} np_n = \lambda > 0 
 \end{align*} биномиальное распределение $ \Binom(n,p_n) $ стремится к распределению Пуассона $ \Poisson(\lambda) $.

 Проверим, что распределение Пуассона определено корректно:
 \begin{align*}
  \sum_{k=0}^{\infty} \frac{\lambda^{k}}{k!}e^{-\lambda} = e^{-\lambda} e^{\lambda} = 1.
 \end{align*} 
\end{exmpl}
\begin{exmpl}[%
геометрическое распределение]
 Пусть есть число $ 0 < p \leqslant 1 $. \textit{Геометрическим распределением} называется называется распределение случайной величины $ \xi \colon\,\Omega \to \left\{ 1,2,3,\ldots \right\} $, для которого
 \begin{align*}
  P(\xi = k) = p(1-p)^{k-1}.
 \end{align*} Обозначение: $ \xi \sim \Geom(p) $.

 Номер шага, на котором случился первый успех в схеме Бернулли с вероятностью успеха $ p $, обладает геометрическим распределением $ \Geom(p) $. Иногда удобнее считать не номер шага первого успеха, а число неудач до первого успеха: тогда из величины будет вычтена единица, и формулы немного поменяются.

 Снова проверим корректность в качестве упражнения:
 \begin{align*}
  \sum_{k=1}^{\infty} p(1-p)^{k-1} = p \cdot \sum_{k=0}^{\infty} (1-p)^{k} = \frac{p}{1 - (1-p)} = 1.
 \end{align*} 
\end{exmpl}

\begin{exmpl}[%
дискретное равномерное распределение]
 \textit{Дискретным равномерным распределением} называется распределение случайной величины $ \xi\colon\,\Omega\to \left\{ 1,2,\ldots,n \right\} $, для которого
 \begin{align*}
  P(\xi = k) = \frac{1}{n}.
 \end{align*}

Общепринятого обозначения нет.
\end{exmpl}

Теперь рассмотрим канонические примеры непрерывных вероятностных распределений.

\begin{exmpl}[непрерывное равномерное распределение]
 \textit{Непрерывным равномерным распределением} называется распределение случайной величины $ \xi\colon\,\Omega \to [a,b] $ с плотностью $ p_\xi \colon\,\R\to\R $, заданной по формуле
 \begin{align*}
  p_\xi(t) = \frac{1}{b-a}\Ind_{[a,b]}(t).
 \end{align*} Обозначение: $ \xi \sim U[a,b] $.
\end{exmpl}
\begin{exmpl}[%
нормальное распределение]
 \textit{Нормальным распределением} (или \textit{распределением Гаусса}) с параметрами $ \mu \in \R $ и $ \sigma > 0 $ называется распределение случайной величины $ \xi \colon\,\Omega \to \R $ с плотностью
 \begin{align*}
  p_\xi(t) = \frac{1}{\sigma \sqrt{2\pi}}e^{-\frac{1}{2} \left( \frac{t-\mu}{\sigma} \right)^{2}}.
 \end{align*} Обозначение: $ \xi \sim \Norm(\mu,\sigma^{2}) $. Корректность проверятся с помощью интеграла Эйлера-Пуассона $ \int_{-\infty}^{+\infty}  e^{-x^{2}} = \sqrt \pi $:
 \begin{align*}
  \frac{1}{\sigma \sqrt{2\pi}} \int\limits_{-\infty}^{+\infty} e^{-\frac{1}{2} \left( \frac{t-\mu}{\sigma} \right)^{2}} \, dt &= \begin{bmatrix}
   x = \frac{t-\mu}{\sigma\sqrt{2}}, & t = \sigma \sqrt 2 x + \mu, & dt = \sigma \sqrt 2 \, dx
  \end{bmatrix} = \\
  &= \frac{1}{\sigma\sqrt{2\pi}} \int\limits_{-\infty}^{+\infty} e^{-x^{2}} \cdot \sigma\sqrt{2}\,dx = \frac{1}{\sqrt \pi} \cdot \sqrt\pi = 1.
 \end{align*} 

 Распределение $ \Norm(0,1) $ называется \textit{стандартным нормальным распределением}. Его плотность равна
 \begin{align*}
  p_\xi = \frac{1}{\sqrt{2\pi}}e^{-t^{2} / 2}.
 \end{align*} 
\end{exmpl}
\begin{exmpl}[%
экспоненциальное распределение]
Пусть $ \lambda > 0 $. \textit{Экспоненциальным распределением} называется распределение случайной величины $ \xi \colon\,\Omega\to[0,+\infty) $ с плотностью
\begin{align*}
 p_\xi(t) = \begin{cases}
  \lambda e^{-\lambda t}, \text{ если } t\geqslant 0;  \\
  0, \text{ иначе}.
 \end{cases} 
\end{align*} Обозначение $ \xi \sim \Exp(\lambda) $.

Снова проверим корректность в качестве упражнения:
\begin{align*}
 \lambda \int\limits_{0}^{+\infty} e^{-\lambda t}\,dt = \lambda \cdot \left. \left( -\frac{1}{\lambda} e^{-\lambda t} \right)\right|_0^{+\infty} = \lambda \cdot \left( 0 - \left( -\frac{1}{\lambda} \right) \right) = 1.
\end{align*} 
\end{exmpl}

\section{Совместное распределение}

Рассмотрим частую ситуацию, когда у нас есть несколько случайных величин, собранных вместе: можно сказать, что у нас есть \textit{случайный вектор}. В таком случае можно говорить как о распределениях каждой случайной величины в отдельности (\textit{распределения координат}), а можно говорить о так называемом \textit{совместном распределении}, которое мы сейчас и определим.

\begin{df}[совместное распределение]
 Пусть есть случайные величины $ \xi_1, \xi_2, \ldots, \xi_n \colon\,\Omega\to\R $. Рассмотрим случайный вектор, составленный из этих величин:
 \begin{align*}
  \vec\xi = (\xi_1, \xi_2, \ldots,\xi_n)\colon\,\Omega \to \R^{n}.
 \end{align*} Тогда \textit{совместным распределением} случайных величин $ \xi_1, \xi_2, \ldots, \xi_n $ называется вероятностная мера $ P_{\vec\xi} \colon\,\B_n \to [0,1] $, которая каждому борелевскому подмножеству $ A \subset \R^{n} $ сопоставляет вероятность того, что точка $ \vec\xi $ попадёт в это множество:
 \begin{align*}
  P_{\vec\xi}(A) = P(\vec\xi \in A).
 \end{align*} 
\end{df}
\begin{remrk*}
 Как и для распределения одной случайной величины, легко видеть, что совместное распределение действительно является вероятностной борелевской мерой на $ \R^{n} $.
\end{remrk*}
\begin{remrk}
 Совместное распределение $ P_{\vec\xi} $ однозначно определяет распределения координат $ P_{\xi_1}, \ldots, P_{\xi_n} $, но не наоборот.
\end{remrk} 
\begin{proof}[\normalfont\textsc{Доказательство}]
 Действительно, 
 \begin{align*}
  P_{\xi_j}(A) = P_{\vec\xi}(A \times \R^{n-1}).
 \end{align*} С другой стороны, пример \ref{example:joint_distribution_cannot_be_deduced_from_coordinate_distributions} показывает, что по распределениям координат нельзя восстановить совместное распределение.
\end{proof}
\begin{exmpl}
 \label{example:joint_distribution_cannot_be_deduced_from_coordinate_distributions}
 Пусть случайные величины $ \xi $ и $ \eta $ принимают значение $ 0 $ или $ 1 $ с вероятностью $ \frac{1}{2} $.
 \begin{itemize}
  \item Предположим, что $ \xi $ и $ \eta $ независимы. Тогда случайный вектор $ (\xi, \eta) $ принимает одно из четырёх значений $ (0,0),(0,1),(1,0),(1,1) $ с равной вероятностью.
  \item Теперь предположим, что $ \xi = \eta $. Тогда случайный вектор $ (\xi,\eta) $ принимает одно из двух значений $ (0,0), (1,1) $ с равной вероятностью.
 \end{itemize}
\end{exmpl}

В примере \ref{example:joint_distribution_cannot_be_deduced_from_coordinate_distributions} мы впервые сказали, что случайные величины могут быть \textit{независимы}. Формализуем это понятие.

\begin{df}[независимость случайных величин]\
 Случайные величины $ \xi_1, \xi_2, \ldots, \xi_n$ называются \textit{независимыми}, если для любых борелевских подмножеств $ A_1, \ldots, A_n \subset \R $ верно
 \begin{align}
  \label{equation:independence_of_random_variables}
  P(\xi_1 \in A_1) \cdot \ldots \cdot P(\xi_n \in A_n) = P(\xi_1 \in A_1, \ldots, \xi_n \in A_n).
 \end{align}
\end{df}
\begin{claim*}
 Случайные величины $ \xi_1, \xi_2, \ldots, \xi_n $ независимы тогда и только тогда, когда для любых борелевских множеств $ A_1, \ldots, A_n \subset \R $ события $ [\xi_1 \in A_1], \ldots, [\xi_n \in A_n] $ независимы.
\end{claim*}
\begin{proof}[\normalfont\textsc{Доказательство}]
 В сторону $(\Longleftarrow)$ очевидно. В сторону $(\Longrightarrow)$: рассмотрим любые борелевские подмножества $ A_{i_1}, \ldots, A_{i_k} \subset \R $, где $ i_1, \ldots, i_k \in [n] $ различны. Для всех индексов $ j \notin \left\{ i_1, \ldots, i_k \right\} $ доопределим $ A_j = \R $, тогда равенство \eqref{equation:independence_of_random_variables} превращается в
 \begin{align*}
  P(\xi_{i_1} \in A_{i_1}) \cdot \ldots \cdot P(\xi_{i_k} \in A_{i_k}) = P(\xi_{i_1} \in A_{i_1}, \ldots, \xi_{i_k} \in A_{i_k}),
 \end{align*} что и требовалось доказать.
\end{proof}

\begin{thm}
 \label{theorem:random_variable_independence_through_distribution_measure_eq}
 Случайные величины $ \xi_1, \ldots, \xi_n $ независимы тогда и только тогда, когда имеет место равенство мер
 \begin{align}
  \label{equation:random_variable_independence_through_distribution_measure_eq}
  P_{\vec\xi} = P_{\xi_1} \times P_{\xi_2} \times \ldots \times P_{\xi_n},
 \end{align} где в правой части стоит произведение мер.
\end{thm}
\begin{proof}[\normalfont\textsc{Доказательство}]
 Пусть величины независимы. По теореме Каратеодори равенство мер достаточно проверять на множествах вида $ A_1 \times \ldots \times A_n $ для некоторых борелевских $ A_1, \ldots, A_n \subset \R $. Проверим:
 \begin{align*}
  P_{\vec\xi}(A_1 \times \ldots \times A_n) &= P(\xi_1 \in A_1, \ldots, \xi_n \in A_n) =  \\
  &= P(\xi_1 \in A_1) \ldots P(\xi_n \in A_n) = \\
  &= P_{\xi_1}(A_1) \ldots P_{\xi_n}(A_n).
 \end{align*} С другой стороны, если есть равенство мер, то все те же равенства также верны.
\end{proof}

Как и для распределений одной случайной величины имеет смысл ввести \textit{функцию распределения} совместных распределений. 

\begin{df}[совместная функция распределения] Пусть $ \vec\xi = (\xi_1, \ldots, \xi_n) $ --- случайный вектор. \textit{Совместной функцией распределения} случайных величин $ \xi_1, \ldots, \xi_n $ называется функция $ F_{\vec\xi} \colon\;\R^{n}\to[0,1] $:
 \begin{align*}
  F_{\vec\xi}(\vec x) := P(\xi_1 \leqslant x_1, \ldots, \xi_n \leqslant x_n).
 \end{align*} 
\end{df}

\begin{remrk*}
 Пусть $ \vec\xi $, $ \vec\eta $ --- случайные вектора из $ n $ величин. По теореме Каратеодори равенство совместных распределений $ P_{\vec\xi} = P_{\vec\eta} $ равносильно равенству совместных функций распределения $ F_{\vec\xi} = F_{\vec\eta} $.
\end{remrk*}

\begin{prop}[cвойства совместной функции распределения]\
 \begin{enumerate}
  \item Функция $ F_{\vec\xi} $ возрастает по каждой координате.
  \item $ \lim\limits_{x_1, \ldots, x_n \to +\infty}  F_{\vec\xi}(\vec x) = 1 $.
 \item $ \lim\limits_{x_1, \ldots, x_n \to -\infty} F_{\vec\xi}(\vec x) = 0 $.
 \item Для всякого $ i \in [n] $ и для всяких $ x_1, \ldots, x_{i-1}, x_{i+1}, \ldots, x_n \in \R $ выполнено
  \begin{align*}
  \lim_{x_i \to +\infty} F_{\vec\xi}(\vec x) = F_{\xi_1, \ldots, \xi_{i-1}, \xi_{i+1}, \ldots, \xi_n}(x_1, \ldots, x_{i-1},x_{i+1},\ldots,x_n).
\end{align*} 
 \end{enumerate}
\end{prop}

\begin{crly}
 \label{corollary:inpendence_joint_F}
 Случайные величины $ \xi_1, \ldots, \xi_n $ независимы тогда и только тогда, когда для любого $ \vec x \in \R^{n} $ верно
 \begin{align}
  \label{equation:random_variable_indepencence_throught_distribtuion_function_eq}
  F_{\vec\xi}(\vec x) = F_{\xi_1}(x_1) \ldots F_{\xi_n}(x_n).
 \end{align} 
\end{crly}
\begin{proof}[\normalfont\textsc{Доказательство}]
 По теореме \ref{theorem:random_variable_independence_through_distribution_measure_eq} достаточно показать, что \eqref{equation:random_variable_indepencence_throught_distribtuion_function_eq} эквивалентно \eqref{equation:random_variable_independence_through_distribution_measure_eq}. Из \eqref{equation:random_variable_independence_through_distribution_measure_eq} в \eqref{equation:random_variable_indepencence_throught_distribtuion_function_eq} очевидно: нужно подставить множества $ A_i = \left(-\infty, x_i\right]   $.

 Докажем теперь в сложную сторону. Будем полагать $ n = 2 $ (для больших $ n $ доказывается абсолютно так же, но записи станут громоздкими, не добавляя понимания), пусть случайные величины обозначены как $ \xi $ и $ \eta $. Равенство мер \eqref{equation:random_variable_independence_through_distribution_measure_eq} достаточно проверить на произвольной паре ячеек $ \left(a, b\right]   $, $ \left(c, d\right] $. Проверим:
 \begin{align*}
  P_\xi \left(a, b\right] \cdot P_\eta \left(c, d\right] &= \left( F_\xi(b) - F_\xi(a) \right) \left( F_\eta(d) - F_\eta(c)  \right) = \\
  &= F_\xi(b)F_\eta(d) - F_\xi(b) F_\eta(c) - F_\xi(a) F_\eta(d) + F_\xi(a) F_\eta(c) = \\
  &= F_{(\xi,\eta)}(b, d) - F_{(\xi,\eta)}(b,c) + F_{(\xi,\eta)}(a,d) + F_{(\xi,\eta)}(a,c) = \\
  &= P_{(\xi,\eta)} \left( \left(a, b\right] \times \left(c, d\right]    \right).
 \end{align*} Последнее равенство верно по формуле включений-исключений (предложение \ref{proposition:inclusion_exclusion_formula}).
\end{proof}

\begin{df}
 Говорят, что случайный вектор $ \vec\xi = (\xi_1, \ldots, \xi_n) $ \textit{абсолютно непрерывен} (совместное распределение $ \xi_1,\ldots,\xi_n $ \textit{абсолютно непрерывно}), если существует неотрицательная измеримая функция $ p_{\vec\xi} \colon\, \R^{n}\to \left[0, +\infty\right)$, называемая \textit{плотностью совместного распределения} (\textit{совместной плотностью}), такая, что для любого $ \vec x \in \R^{n} $ верно
 \begin{align}
  \label{equation:density_of_joint_distribution}
  F_{\vec\xi}(\vec x) = \int\limits_{-\infty}^{x_1}  \ldots \int\limits_{-\infty}^{x_n} p_{\vec\xi}(\vec t) \, dt_n \ldots dt_1.
 \end{align} 
\end{df}
\begin{crly}
 Пусть случайные величины $\xi_1, \ldots, \xi_n $ абсолютно непрерывны. Тогда они независимы, если и только если для любого $ \vec t \in \R^{n} $
 \begin{align}
  \label{equation:independence_density}
  p_{\vec\xi}(\vec t) = p_{\xi_1}(t_1) \ldots p_{\xi_n}(t_n).
 \end{align} В частности, если $ \xi_1, \ldots, \xi_n $ независимы, то случайный вектор $ \vec\xi $ абсолютно непрерывен.
\end{crly}
\begin{proof}[\normalfont\textsc{Доказательство}]
 По следствию \ref{corollary:inpendence_joint_F} достаточно показать, что \eqref{equation:independence_density} эквивалентно \eqref{equation:random_variable_indepencence_throught_distribtuion_function_eq}. Пусть выполнено \eqref{equation:random_variable_indepencence_throught_distribtuion_function_eq}: $ F_{\vec\xi}(\vec x) = F_{\xi_1}(x_1) \ldots F_{\xi_n}(x_n) $. Покажем, что функция $ \vec t \mapsto p_{\xi_1}(t_1) \ldots p_{\xi_n}(t_n) $ подходит под определение плотности. Для любого $ \vec x \in \R^{n} $:
 \begin{align*}
  F_{\xi}(\vec x) &= F_{\xi_1}(x_1) \ldots F_{\xi_n}(x_n) = \left( \int_{-\infty}^{x_1} p_{\xi_1}(t_1)\,dt_1  \right) \ldots \left( \int_{-\infty}^{x_n} p_{\xi_n}(t_n)\,dt_n  \right) = \\
  &= \int\limits_{-\infty}^{x_1} \ldots \int\limits_{-\infty}^{x_n} p_{\xi_1}(t_1) \ldots p_{\xi_n}(t_n)\,dt_n \ldots dt_1.
 \end{align*} Наоборот, если $ p_{\vec\xi}(\vec t) = p_{\xi_1}(t_1) \ldots p_{\xi_n}(t_n) $ --- совместная плотность, то верно то же самое равенство.
\end{proof}

Сделаем ещё одно отступление в теорию меры.

\begin{df}[свёртка мер]
 Пусть есть две конечные борелевские меры $ \mu $ и $ \nu $ на $ \R $. \textit{Свёрткой} мер $ \mu $ и $ \nu $ называется борелевская мера $ \mu \ast \nu $ на $ \R $, определённая следующим образом:
 \begin{align*}
  \mu \ast \nu(A) := \int\limits_{\R} \mu(A - x) \,d\nu(x).
 \end{align*} 
\end{df}
\begin{remrk*}
 Легко видеть, что свёртка мер действительно является мерой:
 \begin{align*}
  \mu\ast\nu \left( \bigsqcup_{k=1}^{\infty}A_k \right) &= \int\limits_{\R} \mu \left( \bigsqcup_{k=1}^{\infty}A_k - x \right) d\nu(x) = \int\limits_{\R} \left( \sum_{k=1}^{\infty} \mu(A_k - x) \right) d\nu(x) = \\
  &= \sum_{k=1}^{\infty} \int\limits_{\R} \mu(A_k - x) \,d\nu(x) = \sum_{k=1}^{\infty}\mu \ast \nu(A_k).
 \end{align*} Знак интеграла и суммы можно было поменять местами по теореме Леви.
\end{remrk*}

Перечислим некоторые свойства свертки мер.

\begin{prop}
 \begin{align}
  \label{equation:measure_convolution_double_integral}
  \mu\ast\nu(A) = \int\limits_{\R^{2}} \Ind_A(x + y)\,d\mu(x)\,d\nu(y).
 \end{align} 
\end{prop}
\begin{proof}[\normalfont\textsc{Доказательство}]
 \begin{align*}
  \mu\ast\nu(A)&=\int\limits_{\R} \mu(A-y)\,d\nu(y) = \int\limits_{\R} \int\limits_{\R} \Ind_{A-y}  (x)\,d\mu(x)\,d\nu(y) = \\
  &= \int\limits_{\R^{2}} \Ind_A(x+y)\,d\mu(x)\,d\nu(y). 
 \end{align*} 
\end{proof}

\begin{prop} Для всяких конечных борелевских мер $ \mu $, $ \nu $ на $ \R $ верно следующее.
 \begin{enumerate}
  \item Есть коммутативность $ \mu \ast \nu = \nu \ast \mu $. Очевидно из формулы \eqref{equation:measure_convolution_double_integral}.
 \item Для конечных борелевских мер $ \mu_1, \ldots, \mu_n $ на $ \R $ верно
\begin{align}
 \label{equation:measure_convolution_many}
 \mu_1 \ast \ldots \ast \mu_n (A) = \int\limits_{\R^{n}}\Ind_A(x_1 + x_2 + \ldots + x_n)\,d\mu_1(x_1) \ldots d\mu_n(x_n).   
\end{align}
\item Ассоциативность: $ \mu_1 \ast (\mu_2 \ast \mu_3) = (\mu_1 \ast \mu_2) \ast \mu_3 $. Очевидно из формулы \eqref{equation:measure_convolution_many}.
\item Линейность: для всякого $ c \geqslant 0 $ верно
 \begin{align*}
  (c\mu) \ast \nu = c \cdot \mu \ast \nu,
 \end{align*} а также для любых конечных борелевских мер $ \mu_1 $, $ \mu_2 $ верно
 \begin{align*}
  (\mu_1 + \mu_2) \ast \nu = \mu_1 \ast\nu + \mu_2\ast\nu.
 \end{align*} 
\end{enumerate}
\end{prop}

\begin{df}[точечная нагрузка]
 Пусть $ x \in \R $ --- точка. Тогда \textit{единичной нагрузкой} (\textit{дельта-мерой Дирака}) в точке $ x $ называется мера
 \begin{align*}
  \delta_x(A) = \begin{cases}
   1, \text{ если } x \in A, \\
   0, \text{ если } x\notin A,
  \end{cases} 
 \end{align*} где $ A \subset \R $.
\end{df}

\begin{prop}
 Для любой конечной борелевской меры $ \mu $ на $ \R $ верно $ \mu \ast \delta_0 = \mu $.
\end{prop}
\begin{proof}[\normalfont\textsc{Доказательство}]
 $ \mu \ast \delta_0(A) = \int_{\R} \mu(A-x)\,d\delta_0(x) = \mu(A).  $
\end{proof}


А что будет, взять свёртку двух мер, имеющих плотность?

\begin{thm}
 \label{theorem:convolution_of_measures_with_density}
 Пусть $ \mu $, $ \nu $ --- конечные борелевские меры на $ \R $ с плотностями $ p_\mu $ и $ p_\nu $ соответственно:
 \begin{align*}
  \mu(A) = \int_{A} p_\mu(t)\,dt, & &\nu(A) = \int_{A} p_\nu(t)\,dt.  
 \end{align*} Тогда мера $ \mu \ast \nu $ имеет плотность:
 \begin{align*}
  \mu \ast \nu(A) = \int_{A} p(t)\,dt, 
 \end{align*} где
 \begin{align}
  \label{equation:density_convolution_functions}
  p(t) = \int_{\R} p_\mu(t - u) \cdot p_\nu(u)\,du.
 \end{align} 
\end{thm}
\begin{df}
 В обозначениях теоремы \ref{theorem:convolution_of_measures_with_density} функция $ p(t) $ называется \textit{свёрткой} функций $ p_\mu(t) $ и $ p_{\nu}(t) $.
\end{df}
\begin{proof}[\normalfont\textsc{Доказательство теоремы \ref{theorem:convolution_of_measures_with_density}}] Рассмотрим функцию $ p(t) $, заданную равенством \eqref{equation:density_convolution_functions} и проверим, что она подходит в качестве плотности меры $ \mu \ast \nu $:
 \begin{align*}
  \int\limits_{A} p(t)\,dt &= \int\limits_{A} \int\limits_{\R}p_\mu(t - u)\cdot p_\nu(u)\,du\,dt = \int\limits_{\R} \int\limits_{\R} \Ind_A(t) \cdot p_\mu(t-u) \cdot p_\nu(u)\,du\,dt.
 \end{align*} Сделав замену переменной $ v = t - u  $ получаем:
 \begin{align*}
  \int\limits_{A} p(t)\,dt = \int\limits_{\R}\int\limits_{\R}\Ind_A(u + v) \cdot p_\mu(v)\cdot p_\nu(u)\,du\,dv.     
 \end{align*} Далее два раза воспользуемся формулой интеграла по мере имеющей плотность (теорема \ref{theorem:integral_on_measure_with_density}):
 \begin{align*}
  \int\limits_{A} p(t)\,dt &= \int\limits_{\R} \left( \int\limits_{\R} \Ind_A(u+v) \cdot p_\nu(u)\,du  \right) p_\mu(v) \, dv = \\ &= \int\limits_{\R} \left( \int\limits_{\R} \Ind_A(u+v)\,d\nu(u)  \right) p_\mu(v)\,dv = \int\limits_{\R}  \int\limits_{\R} \Ind_A(u+v)\,d\nu(u)  \, d\mu(v) = \\
  &= \mu \ast \nu(A).
 \end{align*} 
\end{proof}

\begin{thm}
 \label{theorem:distribution_of_sum_is_measure_convolution}
 Если cлучайные величины $ \xi $ и $ \eta $ независимы, то $ P_{\xi + \eta} = P_\xi \ast P_\eta $.
\end{thm}
\begin{proof}[\normalfont\textsc{Доказательство}]
 Возьмём любое борелевское множество $ A \subset \R $. Рассмотрим следующее множество:
 \begin{align*}
  B = \left\{ (x, y) \in \R^{2} \Mid x + y \in A \right\}.
 \end{align*} Вычислим теперь распределение $ \xi + \eta $:
 \begin{align*}
  P_{\xi+\eta}(A) &= P(\xi + \eta \in A) = P((\xi,\eta)\in B) = P_{(\xi,\eta)}(B) = \int\limits_{\R^{2}} \Ind_B(x,y) \,dP_{(\xi,\eta)}(x,y).
 \end{align*} Раз $ \xi $ и $ \eta $ независимы, то по теореме \ref{theorem:random_variable_independence_through_distribution_measure_eq} имеем $ P_{(\xi,\eta)} = P_\xi \times P_\eta $, из чего следует
 \begin{align*}
  P_{\xi+\eta}(A) &= \int\limits_{\R^{2}} \Ind_B(x,y)\,dP_\xi(x)\,dP_\eta(y) = \int\limits_{\R^{2}} \Ind_A(x + y) \,dP_\xi(x)\, dP_\eta(y) = \xi \ast \eta(A).
 \end{align*} 
\end{proof}

\begin{exmpl}[свёртка с дискретным распределнием]
 Пусть $ \mu $ --- конечная борелевская  мера на $ \R$, а
 \begin{align*}
  \nu = \sum_{k=1}^{\infty} p_k \delta_{x_k}, \qquad p_k \geqslant 0
 \end{align*} --- дискретная мера (считающая мера с коэффициентами). Например, если $ p_1 + p_2 + \ldots = 1 $, то $ \nu $ --- распределение дискретной случайной величины. Тогда
 \begin{align*}
  \mu \ast \nu(A) &= \int\limits_{\R} \mu(A - x) d\nu(x) = \sum_{k=1}^{\infty}p_k \mu(A - x_k).
 \end{align*} 
\end{exmpl}
\begin{exmpl}[свёртка двух распределений Пуассона]
 Пусть $ \xi_1 \sim \Poisson(\lambda_1) $ и $ \xi_2 \sim \Poisson(\lambda_2) $, $ \xi_1 $  и $ \xi_2 $ независимы. Каково распределение $ \xi_1 + \xi_2 $?

 Найдём его, воспользовавшись теоремой \ref{theorem:distribution_of_sum_is_measure_convolution}:
 \begin{align*}
  P_{\xi_1 + \xi_2}(\left\{ n \right\}) &= P_{\xi_1} \ast P_{\xi_2}(\left\{ n \right\}) = \int\limits_{\R^{2}}  \Ind_{\left\{ n \right\}}(x+y)\,dP_{\xi_1}\,dP_{\xi_2} = \\
  &= \sum_{x=0}^{n} \frac{\lambda_1^{x}}{x!} e^{-\lambda_1} \cdot \int\limits_{\R} \Ind_{\left\{ n \right\}}(x+y) \, dP_{\xi_2} = \\
  &= \sum_{x=0}^{n} \frac{\lambda_1^{x}}{x!}e^{-\lambda_1} \cdot \frac{\lambda_2^{n-x}}{(n-x)!}e^{-\lambda_2} = \frac{\left( \lambda_1 + \lambda_2 \right)^{n}}{n!}e^{-(\lambda_1 + \lambda_2)}.
 \end{align*} Таким образом, $ \xi_1 + \xi_2 \sim \Poisson(\lambda_1 + \lambda_2) $.
\end{exmpl}

\begin{exercs*}\
\begin{enumerate}
 \item Пусть $ \xi_1, \xi_2 \sim \Exp(1) $, $ \xi_1 $ и $ \xi_2 $ независимы. Найти распределение $ \xi_1 + \xi_2 $.
 \item Пусть $ \xi_1 \sim \Norm(\mu_1, \sigma_1^{2}) $ и $ \xi_2 \sim \Norm(\mu_2, \sigma_2^{2}) $, $ \xi_1 $ и $ \xi_2 $ независимы. Доказать, что $ \xi_1 + \xi_2 \sim \Norm(\mu_1 + \mu_2, \sigma_1^{2} + \sigma_2^{2}) $.
\end{enumerate}
\end{exercs*}
