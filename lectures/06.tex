% 2023.03.21: lecture 06
\documentclass[../main.tex]{subfiles}
\begin{document}

Геометрический смысл $\cov$. Пусть $ \E \xi^{2} < +\infty $. Тогда
\begin{align*}
 \cov(\xi,\eta) = \E((\xi - \E\xi)(\eta - \xi\eta))
\end{align*} --- почти что скалярное произведение. $ \D\xi = 0 \implies \xi = c $ почти всюду.

Смотрим на классы эквивалентности $ \xi - c = \eta$ почти наверное.

Некоррелированость = ортогональность.

\begin{exmpl}[независимые множества в графе]
 Пусть дан неориентированный граф $ G = (V,E) $, $ n $ вершин и $ \frac{nd}{2} $ рёбер, где $ d \geqslant 1 $. Тогда можно выбрать независимое множество на $ \frac{n}{2d} $ вершинах.
\end{exmpl}
\begin{proof}[\normalfont\textsc{Доказательство}]
 Рассмотрим случайное множество вершин $ S \subset V $ ($ n $ раз подбрасываем нечестную монетку для каждой вершины), это будет наше почти независимое множество.
 \begin{align*}
  P(x \in S) = p,
 \end{align*} Дальше для ребра $ xy \in E $ заведём случайную величину
 \begin{align*}
  \xi_{xy} = \begin{cases}
   1, \text{ если } x, y \in S; \\
   0, \text{ иначе. }
  \end{cases} 
 \end{align*} Тогда $ \E\xi_{xy} = p^{2} $. Рассмотрим случайную величину --- сумму(а по сути количество ребер внутри $S$):
 \begin{align*}
  \xi = \sum_{xy \in E}\xi_{xy}.
 \end{align*} Тогда
 \begin{align*}
  \E\xi = \frac{nd}{2} \cdot p^{2}
 .\end{align*} Для $ x \in V $ рассмотрим
 \begin{align*}
  \eta_x = \begin{cases}
   1, \text{ если } x \in S, \\
   0, \text{ иначе. }
  \end{cases} 
 \end{align*} Тогда $ \E\eta_x = p $.
 \begin{align*}
  \eta = \sum_{x \in V} \eta_x
 \end{align*} $ \E\eta = pn $(ожидаемое количество элементов в $S$ ). Рассмотрим
 \begin{align*}
  \E(\eta - \xi) = np - \frac{nd}{2} \cdot p^{2}.
 \end{align*} Максимизируем по $ p $. Производная:

 \begin{align*}
	 \frac{d}{dp}(np - \frac{nd}{2} \cdot p^{2}) &= n - ndp = 0 \implies p = \frac{1}{d} \implies \\
	 \implies \E(\eta - \xi) &= n \cdot \frac 1 d - \frac{n}{2d} \cdot \frac{1}{d^2} = \frac{n}{2d} \\
 \end{align*}

 А значит найдётся реализация, на которой $ \eta - \xi \geqslant \frac{n}{2d} $. Возьмём это множество $ S $. У каждого ребра выкинем одну из вершин, получим независимое множество из хотя бы $ \frac{n}{2d} $ вершин.
\end{proof}

Рассмотрим следующую теоретико-числовую иллюстрацию.

\begin{exmpl}[]
 Рассмотрим $ \Omega = \left\{ 1,2,\ldots,n \right\} $. Обозначим за $ \nu(k) $ количество различных простых в разложении числа $ k $ на простые. Выберем случайное число $ k $. Какое будет $ \nu(k) $?
\end{exmpl}
\begin{thm}[Харди-Рамануджана]
 Пусть $ \omega(n) \to +\infty $. Тогда
 \begin{align*}
  P( \left| \nu(k) - \log\log n \right| > \omega(n) \sqrt{\log\log n}) \to 0
 \end{align*}  при $ n \to \infty $.
\end{thm}
\begin{proof}[\normalfont\textsc{Доказательство (Туран)}]
 Рассмотрим $ M := \sqrt[10] n $. Пусть
 \begin{align*}
  \xi_p(k) = \begin{cases}
   1, \text{ если } k \ \vdots \ p, \\
   0, \text{ иначе. }
  \end{cases} \\
  \xi = \sum_{p \in \mathbb P, p \leqslant M} \xi_p
 \end{align*} $ \xi(k) $ --- это все различные простые делители $ k $ не превосходящие $M$.
 
 Тогда $ \xi(k) \leqslant \nu(k) \leqslant \xi(k) + 10$. Будем смотреть теперь на $ \xi $.
 \begin{align*}
  \E\xi_p = \frac{\left\lfloor n / p \right\rfloor }{n} = \frac{1}{p} + O\left(\frac{1}{n}\right).
 \end{align*} Тогда
 \begin{align*}
  \E\xi = \sum_{p \in \mathbb P, p \leqslant M} \E\xi_p = \sum_{p \in \mathbb P, p \leqslant M}\left(\frac{1}{p} + O \left( \frac{1}{n} \right) \right) = \log\log M + O(1) = \log\log n + O(1).
 \end{align*} Здесь пользовались фактом, что
 \begin{align*}
  \sum_{p \in \mathbb P, p \leqslant M}  \frac{1}{p} = \log \log M + O(1).
 \end{align*}
 В одну сторону
 \begin{align*}
  \prod_{p \in \mathbb P, p \leqslant k} \frac{1}{1 - \frac{1}{p}} = \prod_{p \in \mathbb P, p \leqslant k} \sum_{j=1}^{\infty} \frac{1}{p^{j}} \geqslant 1 + \frac{1}{2} + \ldots + \frac{1}{k} \approx \log k.
 \end{align*} В другую
 \begin{align*}
  \sum_{n \leqslant p \leqslant n^{2}, p \in \mathbb P} \frac{1}{p} < \frac{4}{3}.
 \end{align*} 

 Продолжим.
 \begin{align*}
  \D\xi = \sum_{p \in \mathbb P,\; p \leqslant M} \D\xi_p + 2 \sum_{p,q \in \mathbb P,p < q \leqslant M} \cov(\xi_p,\xi_q).
 \end{align*}
 \begin{align*}
	 \D\xi_p = \underbrace{\E\xi_p^{2}}_{=\E \xi_p} - (\E\xi_p)^{2} = \frac{1}{p} + O \left( \frac{1}{n} \right) - \left(\frac{1}{p} + O\left(\frac{1}{n}\right)\right)^{2} = \frac{1}{p} - \frac{1}{p^{2}} + O \left( \frac{1}{n} \right).
 \end{align*} Тогда
 \begin{align*}
  \sum_{p \in \mathbb P, p \leqslant M} \D\xi_p = \sum \frac{1}{p} - \sum \frac{1}{p^{2}} + \sum O \left( \frac{1}{n} \right) = \log\log n + O(1).
 \end{align*}
 \begin{align*}
	 \cov(\xi_p, \xi_q) = \E(\xi_p \cdot \xi_q) - \E\xi_p \cdot \E\xi_q &= \\
  = \frac{\left\lfloor \frac{n}{pq} \right\rfloor }{n} - \frac{\left\lfloor \frac{n}{p} \right\rfloor}{n} \cdot \frac{\left\lfloor \frac{n}{q} \right\rfloor }{n} \geqslant \frac{\frac{n}{p \cdot q} - 1}{n} - \frac{\frac{n}{p }}{n } \cdot \frac{\frac{n}{p }}{n} &\geqslant -\frac{1}{n} \\
  = \frac{\left\lfloor \frac{n}{pq} \right\rfloor }{n} - \frac{\left\lfloor \frac{n}{p} \right\rfloor}{n} \cdot \frac{\left\lfloor \frac{n}{q} \right\rfloor }{n} \leqslant \frac{\frac{n}{p \cdot q}}{n} - \frac{\frac{n}{p} - 1}{n} \cdot \frac{\frac{n}{q} - 1}{n} &\leqslant \frac{1}{n} \left(\frac{1}{p} + \frac{1}{q}\right)
 \end{align*} 
 \begin{align*}
	 \sum_{p < q \leqslant M} \cov(\xi_p, \xi_q) &= \sum_{p < q \leqslant M} \frac 1 n \left(\frac 1 p + \frac 1 q\right) \geqslant -\frac{M^2}{n} = O(1) \\
	 \frac 1 n \sum_{p < q \leqslant M} \left(\frac 1 p + \frac 1 q\right) &= \frac 1 2 \cdot \frac 1 n \sum_{p \neq q \leqslant M} \left(\frac 1 p + \frac 1 q\right) \leqslant \frac 1 2 \frac 1 n \cdot 2M \sum_{p \leq M} \frac 1 p = O(1)
 \end{align*}
  А значит $ \D\xi = \log\log n + O(1) $.

  Выполнено вложение:
  \begin{align*}
   \left\{\left| \nu(k) - \log\log n \right| > \omega(n) \sqrt{\log\log n} \right\} \subset \left\{ \left|\xi(k) - \E\xi \right| > \omega(n)\sqrt{\log\log n} - C \right\}.
  \end{align*} Воспользуемся неравенством Чебышёва:
  \begin{align*}
	P( \left| \xi - \E\xi \right| > \omega(n)\sqrt{\log\log n} - C) &\leqslant \frac{\D\xi}{\left( \omega(n)\sqrt{\log \log n} - C \right)^{2}} = \\
   &= \frac{\log\log n + O(1)}{(\omega(n)\sqrt{\log \log n} - C)^{2}} \to 0.
  \end{align*} 
\end{proof}
\begin{remrk}
 На самом деле верно
 \begin{align*}
  \lim_{n \to \infty} \frac{1}{n} \cdot \# \left\{ k \Mid a \leqslant \frac{\nu(k)- \log\log n}{\sqrt{\log \log n}} \leqslant b \right\} = \frac{1}{\sqrt{2\pi}} \int\limits_{a}^{b} e^{-t^{2} / 2}\,dt.
 \end{align*} Доказательство ровно такое же, нужно сослаться только не на Чебышёва, а на ЦПТ.
\end{remrk}

\begin{remrk*}[ещё геометрический комментарий]
 Пусть. $ \E\xi^{2} < +\infty $ Рассмотрим скалярное произведение
 \begin{align*}
  \left\langle \xi,\eta \right\rangle := \E(\xi\eta)
 \end{align*} Это $ L^{2}(\Omega, P) $. $\E\xi$ --- это проекция на прямую констант. Нужно проверить, что
 \begin{align*}
  \left\langle \xi - \E\xi, c \right\rangle = 0,
 \end{align*} а это тривиально.
\end{remrk*}

\section{Сходимость случайных величин.}

Докажем техническую, но полезную саму по себе теорему.

\begin{thm}
 Пусть $ \xi_1, \xi_2, \xi_3, \ldots $ --- независимые случайные величины. Пусть есть много функций $ f_i \colon\;\R^{n_i} \to \R $, измеримых по Борелю. Тогда случайные величины
 \begin{align*}
  f_1(\xi_1, \ldots, \xi_{n_1}),\; f_2(\xi_{n_1 + 1}, \ldots, \xi_{n_1 + n_2}),\; f_3(\xi_{n_1 + n_2 + 1}, \ldots, \xi_{n_1 + n_2 + n_3}),\; \ldots
 \end{align*} должны быть независимы.
\end{thm}
\begin{proof}[\normalfont\textsc{Доказательство}]
 Доказываем для $ \xi_1, \ldots, \xi_m, \eta_1, \ldots \eta_n $. Тогда
 \begin{align*}
  f(\xi_1, \ldots, \xi_m),\;g(\eta_1, \ldots, \eta_n)
\end{align*}  независимы. Нужно доказать, что события $ \left \{ f(\xi_1, \ldots,\xi_m) \in \tilde A \right \} $ и $ \left\{ g(\eta_1, \ldots, \eta_n) \in \tilde B \right\} $ независимы. Но левое событие, это $ \vec\xi \in f^{-1}(\tilde A) $, а правое событие --- это $ \vec\eta \in g^{-1}(\tilde B) $. Прообразы --- борелевские множества, поэтому достаточно доказать независимость событий $ \vec\xi \in A $ и $ \vec\eta \in B $ для любых борелевских множеств $ A \subset\R^{m} $ и $ B \subset \R^{n} $. На самом деле, достаточно ограничится ячейками: счётные дизъюнктные объединения ячеек порождают все борелевские множества. Нужно проверить
 \begin{align*}
  P(\vec\xi \in A) P(\vec\eta \in B) = P(\vec\xi \in A, \vec\eta \in B)
 \end{align*} можно суммировать по $ A $ и по $ B $. А на ячейках очевидно:
 \begin{align*}
	 P(\vec\xi \in \left(a, b\right]  ) = P(\xi_1 \in \left(a_1, b_1\right], \ldots, \xi_m \in \left(a_m, b_m\right]    ) = \prod_{i \in [m]}P(\xi_i \in (a_i, b_i])
 \end{align*} Все понятно.
\end{proof}

\begin{df}\
 \begin{enumerate}
  \item $ \xi_n $ сходится к $ \xi $ почти наверное (с вероятностью $ 1 $), если
   \begin{align*}
    P \left( \omega \in \Omega : \lim_{n \to \infty} \xi_n(\omega) = \xi(\omega) \right) = 1.
   \end{align*} 
  \item $ \xi_n $ сходится к $ \xi $ в среднем порядка $ r \geq 1 $, если
   \begin{align*}
    \E \left| \xi_n - \xi \right|^{r} \to 0.
   \end{align*} На самом деле, это сходимость в пространстве Лебега $ L^{r}(\Omega,P) $.
  \item $ \xi_n $ сходится к $ \xi $ по вероятности, если для любого $ \eps > 0 $
   \begin{align*}
    P(\left| \xi_n - \xi \right| > \eps) \to 0
   \end{align*} при $ n \to \infty $. Обозначение $ \xi_n \xrightarrow{P} \xi $.
  \item $ \xi_n : \Omega_n \to \R$ сходится к $ \xi : \Omega \to R$ по распределению, если
   \begin{align*}
    \lim_{n \to \infty} F_{\xi_n}(x) = F_\xi(x)
   \end{align*} для любой точки $ x $, в которой $ F_\xi $ непрерывна.
 \end{enumerate}
\end{df}

Связь между сходимостями.

\begin{itemize}
	

	\item $ 1 \implies 3 $ было в теории меры.

	\item $ 2 \implies 3 $: тут просто.
\begin{align*}
 P(\left| \xi_n - \xi \right| > \eps) \leqslant \frac{\E \left| \xi_n - \xi \right|^{r}}{\eps^{r}} \to 0.
\end{align*}

\item $ 1 \not \implies 2 $. (и тогда $ 3 \not \implies 2 $). Берём $ \Omega = [0,1] $ с мерой Лебега. $ \xi_n = n^{\frac{1}{r}} \Ind_{[0, \frac{1}{n}]} $. $ \xi_n \to 0 $ почти наверное. Но
\begin{align*}
 \E \xi_n^{r} = n \E\Ind_{[0, \frac{1}{n}]} = 1 \not\rightarrow 0.
\end{align*} 

\item Нет $ 2 \not \implies 1 $ (и тогда нет $ 3 \not \implies 1 $).

Берём $ \Omega = [0,1] $ с мерой Лебега. Берём $ \Ind_{[0, \frac{1}{2}]} $, $ \Ind_{[\frac{1}{2}, 1]} $, $ \Ind_{[0, \frac{1}{4}]} $, \ldots. Не стремится ни к чему почти наверное, но при это очевидно, что
\begin{align*}
 \E \xi^{r}_n = \E \xi_n = \text{длина отрезка} \to 0.
\end{align*}

\item Зато есть стрелка из $ 3 \implies 4 $. Cобытие $ \left\{ \xi_n > x \right\} \supset \left\{ \xi > x + \eps \right\} \cap \left\{ \left| \xi_n - \xi \right| < \eps \right\}$. Тогда для отрицаний
\begin{align*}
 \left\{ \xi_n \leqslant x \right\} \subset \left\{ \xi \leqslant x + \eps \right\} \cup \left\{ \left| \xi_n - \xi \right| \geqslant \eps \right\}.
\end{align*} Пририсуем вероятности:
\begin{align*}
 P(\xi_n \leqslant x) \leqslant P(\xi \leqslant x + \eps) + P(\left| x_n - \xi \right| \geqslant \eps).
\end{align*} При $ \eps \to 0 $ и $ n \to \infty $ имеем $ \varlimsup\limits_{n \to \infty} P(\xi_n \leqslant x) \leqslant P(\xi \leqslant x)   $.
\begin{align*}
 \varlimsup\limits_{n \to \infty} F_{\xi_n} (x) \leqslant F_{\xi}(x)
.\end{align*} В другую сторону:
\begin{align*}
	\left\{ \xi_n \leqslant x \right\} &\supset \left\{ \xi \leqslant x - \eps \right\} \cap \left\{ \left| \xi_n - \xi \right| < \eps \right\} \\
	\left\{ \xi_n > x \right\} &\subset \left\{ \xi > x - \eps \right\} \cup \left\{ \left| \xi_n-\xi \right| \geqslant \eps \right\} \\
	P(\xi_n > x) &\leqslant P(\xi > x - \eps) + P(\left| \xi_n - \xi \right| \geqslant \eps) \\
	1 - F_{\xi_n}(x) &\leqslant 1 - F_{\xi}(x - \eps) + P(\left| \xi_n - \xi \right| \geqslant \eps) \\
	F_{\xi_n}(x) &\geqslant F_\xi(x - \eps) - P(\left|  \xi_n - \xi \right| \geqslant \eps) \\
	\varliminf_{n \to \infty} F_{\xi_n}(x) &\geqslant F_{\xi}(x - \eps)
\end{align*} Получаем $ \lim_{n \to \infty} F_{\xi_n}(x) = F_{\xi}(x) $, если устремить $ \eps \to 0 $.

\item $ 4 \not \implies 3 $ --- неправильная постановка вопроса, но можно придумать пример на одном пространстве (упражнение).

\end{itemize}
\end{document}

