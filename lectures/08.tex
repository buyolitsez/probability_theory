% 2023.04.04 lecture 08
\documentclass[../main.tex]{subfiles}
\begin{document}
\begin{exmpl}\
\begin{enumerate}
  \item Найдём характеристическую функцию нормально распределённой случайной величины $ \xi \sim \Norm(0,1) $:
   \begin{align*}
    \phi_\xi(t) &= \E (e^{it\xi} ) = \frac{1}{\sqrt{2\pi}} \int\limits_{\R} e^{itx}  \cdot e^{-x^{2} / 2}\,dx = \frac{e^{-t^{2} / 2}}{\sqrt{2\pi} } \underbrace{\int\limits_{\R} e^{-(x-it)^{2} / 2} \,dx}_{I} = 
   \end{align*}
\begin{figure}[ht]
    \centering
	\incfig[0.5]{characteristic_function_normall}
	\caption{Рисунок контура по которому интегрируем.}
    \label{fig:characteristic_function_normall}
\end{figure}
   \begin{align*}
	I = \int\limits_{\R} e^{-(x-it)^{2} / 2} \,dx = \lim_{R \to \infty} \int\limits_{-R-it}^{R-it} e^{-z^{2} / 2}\,dz
   \end{align*} Здесь комплексный анализ
   \begin{align*}
	0 = \varointctrclockwise \limits_{\Gamma_R}  e^{-z^{2} / 2}\,dz = \underbrace{\int\limits_{-R-it}^{R-it}}_{\to I}  + \underbrace{\int\limits_{R-it}^{R}}_{\to 0}  + \underbrace{\int\limits_{R}^{-R}}_{\to -\sqrt{2\pi}} + \underbrace{\int\limits_{-R}^{-R-it}}_{\to 0}
   \end{align*} Получаем
   \begin{align*}
    I = \sqrt{2\pi}.
   \end{align*} Тогда
   \begin{align*}
    \varphi_\xi(t) = e^{-t^{2} / 2}.
   \end{align*}
  \item Если $ \eta \sim \Norm(a, \sigma^{2}) $. Тогда $ \eta = a + \sigma \xi $, где $ \xi \sim \Norm(0,1) $. Значит
   \begin{align*}
    \varphi_\eta(t) = \varphi_{a + \sigma \xi}(t) = e^{iat} \varphi_\xi(\sigma t) = e^{iat - \sigma^{2}t^{2} / 2}.
   \end{align*}
 \end{enumerate}
\end{exmpl}

\begin{thm}
 Если $ \E \left| \xi \right|^{n} < +\infty $, то
 \begin{align*}
  \varphi_\xi^{(k)}(t) = \E((i\xi)^{k}e^{it\xi})
 \end{align*} при $ k = 0, 1, \ldots, n $. В частности,
 \begin{align*}
  \varphi_\xi^{(k)}(0) = i^{k} \E\xi^{k}.
 \end{align*}
\end{thm}
\begin{proof}[\normalfont\textsc{Доказательство}]
 Индукция по $ k $. База --- определение характеристической функции.

 Переход $ k \mapsto k + 1 $.
 \begin{align*}
  \varphi_\xi^{(k+1)}(t) &= \lim_{h \to 0} \frac{\varphi_\xi^{(k)}(t+h) - \varphi_\xi^{(k)}(t)}{h} = \lim_{h \to 0} \frac{\E((i\xi)^{k}e^{i(t+h)\xi}) - \E((i\xi)^{k}e^{it\xi})}{h} = \\
  &= \lim_{h \to 0} \E \frac{(i\xi)^{k}e^{it\xi} \cdot (e^{ih\xi} - 1)}{h}.
 \end{align*} Осталось понять, что
 \begin{align*}
  \lim_{h \to 0} \int\limits_{\R} (ix)^{k}e^{itx} \frac{e^{ihx} - 1}{h}\,dP_\xi(x) = \int\limits_{\R} \lim_{h \to 0} (\ldots).
 \end{align*} Теорема Лебега о мажорируемой сходимости. Соорудим мажоранту:
 \begin{align*}
  \left| (ix)^{k}e^{itx}\frac{e^{ihx}-1}{h} \right| = \left| x \right|^{k} \left| \frac{e^{ihx}-1}{h} \right|.
 \end{align*} Если $ \left| h \right| \geqslant \frac{1}{\left| x \right|} $, то
 \begin{align*}
  \left| \frac{e^{ihx}-1}{h} \right| \leqslant \frac{2}{\left| h \right|} \leqslant 2 \left| x \right|.
 \end{align*} Если же $ \left| h \right| \leqslant \frac{1}{\left| x \right|} $, то
 \begin{align*}
	 \left| \frac{e^{ihx}-1}{h} \right| = \left| \frac{1 + O(ihx) - 1}{h} \right| = O(x) \leqslant C \left| x \right|.
 \end{align*} так как функция непрерывна от $ hx $, задана на компакте. Так или иначе
 \begin{align*}
  \left| (ix)^{k}e^{itx}\frac{e^{ihx}-1}{h} \right| \leqslant C \left| x \right|^{k+1}. 
 \end{align*} Но функция $ \left| x \right|^{k+1} $ суммируема, так как $ \E \left| \xi \right|^{k+1}  < +\infty$. Поэтому теорема Лебега о мажорируемой сходимости работает.
\end{proof}
\begin{crly}
 \begin{align*}
  \E\xi = -i \varphi'_\xi(0)
 \end{align*}
 \begin{align*}
  \D\xi = -\varphi''_\xi(0) + \left( \varphi'_\xi(0) \right)^{2}
 \end{align*}
\end{crly}

Эта теорема частично работает и в другую сторону.

\begin{thm}
 Если существует конечная вторая производная $ \varphi''_\xi(0) $, то $ \E\xi^{2} < +\infty $.
\end{thm}
\begin{remrk*}
 В общем виде теорема выглядит так: если существует конечная $ \varphi^{(2n)}_\xi(0) $, то $ \E\xi^{2n} < +\infty $.
\end{remrk*}
\begin{proof}[\normalfont\textsc{Доказательство}]
 \begin{align*}
  \E\xi^{2} &= \int\limits_{\R} x^{2}\,dP_\xi(x) = \int\limits_{\R} \lim_{t \to 0} \left(\frac{\sin(tx)}{t} \right)  ^{2}dP_\xi(x)
 \end{align*} По лемме Фату
 \begin{align*}
  \leqslant \varliminf_{t \to 0} \int\limits_{\R} \left( \frac{\sin(tx)}{t} \right)^{2}\,dP_\xi(x)
 \end{align*} Но
 \begin{align*}
  (\sin(tx))^{2} = \left( \frac{e^{itx} - e^{-itx}}{2i} \right)^{2} = -\frac{e^{2itx} + e^{-2itx}-2}{4}.
 \end{align*} Тогда
 \begin{align*}
  &\leqslant \varliminf_{t \to 0} \frac{1}{4t^{2}} \int\limits_{\R} \left( 2-e^{2itx} -e^{-2itx} \right) \,dP_\xi(x) = \\
  &= \varliminf_{t\to 0} \frac{1}{4t^{2}} \left( 2 - \varphi_\xi(2t) - \varphi_\xi(-2t) \right) = \varliminf_{ t\to 0} \frac{2  - \varphi_\xi(2t) - \varphi_\xi(-2t)}{4t^{2}}.
 \end{align*}
 Тейлор в нуле
 \begin{align*}
  \varphi_\xi(s) = 1 + \varphi'_\xi(0) \cdot s + \frac{\varphi''_\xi(0)}{2}s^{2} + o(s^{2}).
 \end{align*} Тогда
 \begin{align*}
  \varphi_\xi(2t) + \varphi_\xi(-2t) = 2 + \varphi''_\xi(0) \cdot 4t^{2} + o(t^{2}).
 \end{align*}
 Тогда
 \begin{align*}
  \varliminf_{t \to 0} (-\varphi''_\xi(0) + o(1)) = -\varphi''_\xi(0)
 \end{align*} Общая ситуация доказывается точно так же, только много писать. Вторая производная в нуле всегда отрицательна.
\end{proof}

\begin{thm}[формула обращения]
 Пусть $ P_\xi(\left\{ a \right\}) = P_\xi(\left\{ b \right\}) = 0 $. Тогда
 \begin{align*}
  P(a \leqslant \xi \leqslant b) = \lim_{T \to +\infty} \frac{1}{2\pi} \int\limits_{-T}^{T}  \frac{e^{-iat}-e^{-ibt}}{it} \varphi_\xi(t)\,dt.
 \end{align*}
 \begin{align*}
  = \mathrm{p.v.} \frac{1}{2\pi} \int\limits_{\R} \frac{e^{-iat}-e^{-ibt}}{it} \varphi_\xi(t)\,dt.
 \end{align*}
\end{thm}
\begin{proof}[\normalfont\textsc{Доказательство}]
 Рассмотрим случай, когда $ a = -1 $, $ b = 1 $. Нужно проверить
 \begin{align*}
  P(-1 \leqslant \xi \leqslant 1) = \lim_{T \to +\infty} \frac{1}{2\pi}\int\limits_{-T}^{T} \frac{e^{it}-e^{-it}}{it}\varphi_\xi(t)\,dt.
 \end{align*}
 По теореме Фубини:
 \begin{align*}
  \int\limits_{-T}^{T} \frac{e^{it}-e^{-it}}{it}\varphi_\xi(t)\,dt &= \int\limits_{-T}^{T} \frac{e^{it}-e^{-it}}{it} \int\limits_{\R} e^{itx}\,dP_\xi(x)\,dt = \\
  &= \int\limits_{\R} \int\limits_{-T}^{T} e^{itx} \cdot \frac{e^{it}-e^{-it}}{it}dt\,dP_\xi(x). 
 \end{align*} Модуль ограничен:
 \begin{align*}
  \left| \frac{e^{it}-e^{it}}{it} \right| \leqslant C
 \end{align*} всё суммируемое и мера $dP_\xi$ конечна . Далее обозначим
 \begin{align*}
  \Phi_T(x) &= \int\limits_{-T}^{T} e^{itx} \cdot \underbrace{\frac{e^{it}-e^{-it}}{it}}_{\int_{-1}^{1} e^{itu}\,du}\,dt = \\
  &= \int\limits_{-T}^{T} e^{itx}\int\limits_{-1}^{1} e^{itu}\,du\,dt = \int\limits_{-1}^{1} \int\limits_{-T}^{T} e^{it(x+u)}\,dt\,du = \\
  &= \int\limits_{-1}^{1} \frac{e^{iT(x+u)} - e^{-iT(x+u)}}{i(x+u)}\,du = \\
  &= \int\limits_{-1}^{1} \frac{2\sin(T(x+u))}{x+u}\,du = \begin{bmatrix}
   v = T(x+u), & dv = T\,du
  \end{bmatrix} = \\
  &= \int\limits_{T(x-1)}^{T(x+1)} \frac{2\sin v}{v}\,dv.
 \end{align*} Предположим, что
 \begin{align*}
  \lim_{T \to \infty} \int\limits_{\R} \Phi_T(x)\,dP_\xi(x) = \int\limits_{\R} \lim_{T \to +\infty}   \Phi_T(x)\,dP_\xi(x)
 \end{align*} Тогда
 \begin{align*}
  \lim_{T \to +\infty}  \Phi_T(x) = \lim_{T \to +\infty} \int\limits_{T(x-1)}^{T(x+1)}  \frac{2\sin v}{v}\,dv 
 \end{align*} 

 \begin{itemize}
	\item $x > 1 \implies T(x + 1), T(x - 1) \to +\infty$, а так как интеграл сходящийся, то стремление может быть только к нулю. 
	\item $-1 < x < 1 \implies \int_{T(x - 1)}^{T(x + 1)} \frac{\sin v}{dv} \, dv \to \int_{-\infty}^{+\infty} \frac{2 \sin v}{v}\,dv = 4 \int_0^{+\infty} \frac{\sin v}{v} \, dv = 2\pi$ 
	\item $x = 1  \implies \int_{T(X - 1)}^{T(X + 1)} \frac{2 \sin v}{v}\,dv \to \int_0^{+\infty} \frac{2 \sin v}{v} \, dv = \pi$
	\item $x < -1 $ и $x = -1$  аналогично. 
 \end{itemize}
	
 \begin{align*}
 	= \begin{cases}    0, \text{ если } x > 1, x < -1  \\    \pi, \text{ если } x = 1, x = -1 \\    2\pi, \text{ если } -1 < x < 1   \end{cases}
 	= 2\pi \cdot \Ind_{[-1,1]}(x) 
 \end{align*} почти всюду, т. к. $P_{\xi}(\{a\}) = P_{\xi}(\{b\}) = 0$. Тогда

 \begin{align*}
  = \int\limits_{\R} 2\pi \Ind_{[-1,1]} (x)\,dP_\xi(x) = 2\pi \cdot P_\xi [-1, 1].
 \end{align*} Осталось обосновать перестановку предела и интеграла. Теорема Лебега о мажорируемой сходимости. Мажоранта:
 \begin{align*}
  F(y) = \int\limits_{-\infty}^{y} \frac{2\sin v}{v}\,dv. 
 \end{align*}

  Как свести общий случай к $ a = -1 $, $ b = 1 $? Пусть \begin{align*}
   \xi = \frac{a+b}{2} + \frac{b-a}{2}\cdot\eta
  \end{align*} Тогда
  \begin{align*}
  \left\{ a \leqslant \xi \leqslant b \right\} \Longleftrightarrow \left\{ -1 \leqslant \eta \leqslant 1 \right\}.
  \end{align*} Значит,
  \begin{align*}
   P(a \leqslant \xi \leqslant b) &= P(-1 \leqslant \eta \leqslant 1) = \lim_{T \to +\infty} \frac{1}{2\pi} \int\limits_{-T}^{T} \frac{e^{it}-e^{-it}}{it}\varphi_\eta(t)\,dt = \\
	 & \overset{?}{=} \lim_{T \to +\infty} \frac{1}{2\pi} \int\limits_{-T}^{T} \frac{e^{-iat} - e^{-ibt}}{it}\varphi_\xi(t)\,dt 
  \end{align*}

	Покажем, что на месте $\overset{?}{=}$ действительно есть равенство. Для этого раскроем $\varphi_{\xi} $ по правилу $\varphi_\xi(t) = e^{i \frac{a+b}{2}t}\varphi_\eta\left(\frac{b-a}{2}t\right)$ и выплнем преобразования:	

	\begin{align*}
		\int\limits_{-T}^{T} \frac{e^{-iat} - e^{ibt}}{it} \varphi_{\xi}(t)\,dt &= \int\limits_{-T}^{T} \frac{e^{-iat}-e^{-ibt}}{it}e^{i \frac{a+b}{2} t} \cdot \varphi_\eta \left( \frac{b-a}{2}t \right)\,dt = \\
   &= \int\limits_{-T}^{T} \frac{e^{i \frac{b-a}{2}t} -e^{-i \frac{b-a}{2} t}}{it} \varphi_\eta \left( \frac{b-a}{2}t \right)dt = \\
   &= \left[s = \frac{b - a}{2} t\right] = \\
   &= \int\limits_{- \frac{b-a}{2}T}^{\frac{b-a}{2}T} \frac{e^{is}-e^{-is}}{is}\varphi_\eta(s)\,ds 
	\end{align*}
	Устремляем $T$ в бесконечность и получаем, что равенство $\overset{?}{=}$ верно.
\end{proof}

Сделаем выводы из <<первой навороченной теоремы про характеристические функции>>.

\begin{crly}
 Если $ \varphi_\xi = \varphi_\eta $, то $ P_\xi = P_\eta $.
\end{crly}
\begin{proof}[\normalfont\textsc{Доказательство}]
 Обозначим
 \begin{align*}
 A = \left\{ x \in \R \Mid P_\xi(\left\{ x \right\}) > 0 \right\}
\end{align*} Тогда множество $ A $ не более, чем счётное (счётное объединение конечных множеств $\{x \in \R \mid P_{\xi}(\{x\}) > \frac{1}{n}\}$). Значит, почти все точки хорошие. Мы знаем, что $ \varphi_\xi $ однозначно определяет вероятности
 \begin{align*}
  P_\xi([a,b]) = P_\eta( \left[a, b\right)),
 \end{align*} если $ a,b \notin A $.

 Пусть $ a \in A $.  Возьмём  $ a_n \notin A $ такие, что  $ a_n $ убывают к  $ a $ . Знаем, что
 \begin{align*}
  \left(a, b\right]  = \bigcup_{n=1}^{\infty}[a_n,b].
 \end{align*} Множества возрастают, по непрерывности меры сверху всё ок:
 \begin{align*}
  P_\xi \left(a, b\right] = \lim_{n \to \infty} P_\xi [a_n,b] 
 \end{align*} --- однозначно определяется.

 Пусть $ b \in A $. Возьмём  $ b_n \notin A $,  $ b_n $ убывают к  $ b $. Тогда здесь
 \begin{align*}
  \left(a, b\right] = \bigcap_{n=1}^{\infty} \left(a, b_n\right] ,
 \end{align*} значит
 \begin{align*}
  P_\xi \left(a, b\right] = \lim P_\xi \left(a, b_n\right]   
 \end{align*} --- определяется однозначно. Меры совпадают на ячейках $  \implies $ совпадают всюду.
 
\end{proof}
\begin{crly}
 Если $ \int_{\R} \left| \varphi_\xi(t) \right|\,dt < +\infty  $, то $ F_\xi $ имеет плотность
 \begin{align*}
  p_\xi(x) = \frac{1}{2\pi} \int\limits_{\R} e^{-itx}\varphi_\xi(t)\,dt. 
 \end{align*} Это преобразование Фурье функции $\varphi_\xi$ .
\end{crly}
\begin{proof}[\normalfont\textsc{Доказательство}]
 Из абсолютной сходимости интеграла $ \int_{\R} \left| \varphi_\xi(t) \right|\,dt  $ следует что
 \begin{align*}
 \int\limits_{\R} \underbrace{\frac{e^{-iat}-e^{-ibt}}{it}}_{\text{огр.}} \varphi_\xi(t)\,dt
 \end{align*} абсолютно сходится. Тогда
 \begin{align*}
  P(a \leqslant \xi \leqslant b) = \frac{1}{2\pi} \int\limits_{\R} \frac{e^{-iat}-e^{ibt}}{it}\varphi_\xi(t)\,dt. 
 \end{align*} Нужно показать
 \begin{align*}
  P(a \leqslant \xi \leqslant b) = \int\limits_{a}^{b} p_\xi(x)\,dx.
 \end{align*} Проверим
 \begin{align*}
  \int\limits_{a}^{b} \frac{1}{2\pi} \int\limits_{\R} e^{-itx}\varphi_\xi(t)\,dt\,dx &= \frac{1}{2\pi}\int\limits_{\R}  \int\limits_{a}^{b} e^{-itx}\,dx\,\varphi_\xi(t)\,dt = \\
  &= \frac{1}{2\pi} \int\limits_{\R} \frac{e^{-iat}-e^{-ibt}}{it} \varphi_\xi(t)\,dt.
 \end{align*} Всё получилось!
\end{proof}

\begin{thm}
 Пусть $ \xi_k \sim \Norm(a_k, \sigma_k^{2}) $ независимы. Пусть $ c_1, c_2, \ldots, c_n $ не все нули. Тогда
 \begin{align*}
  \xi = a_0 + c_1 \xi_1 + \ldots + c_n \xi_n \sim \Norm(a, \sigma^{2}),
 \end{align*} где
 \begin{align*}
  a = a_0 + \sum_{k=1}^{n}c_ka_k
 \end{align*} и
 \begin{align*}
  \sigma^{2} = \sum_{k=1}^{n}c_k^{2}\sigma_k^{2}.
 \end{align*}
\end{thm}
\begin{proof}[\normalfont\textsc{Доказательство}]
 Посчитаем хар. функцию $ \xi $:
 \begin{align*}
  \varphi_\xi(t) = \varphi_{a_0 + \sum_{k=1}^{n}c_k\xi_k}(t) = e^{ia_0 t} \prod_{k=1}^{n}\varphi_{\xi_k}(c_k t) = e^{ia_0t}\prod_{k=1}^{n} e^{-c_k^{2}\sigma_k^{2} t^{2} / 2} \cdot e^{ic_ka_k t} \implies \text{ ОК }
 \end{align*}
\end{proof}

\end{document}

