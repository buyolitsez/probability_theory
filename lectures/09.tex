% 2023.04.11 lecture 09
\documentclass[../main.tex]{subfiles}
\begin{document}
\section{Сходимость по распределению.}
\begin{df}
 Последовательность случайных величин $ \xi_1, \xi_2, \ldots $ \textit{сходится к $ \xi $ по распределению}, если $ F_{\xi_n}(x) \to F_\xi(x) $ для любой точки непрерывности $ x $ функции $ F_\xi $.
\end{df}
Почти всю лекцию мы будем пытаться понять, чему равносильна сходимость по распределению.
\begin{remrk}
  Пусть cлучайные величины $ \xi $ и $ \eta $ независимы, $ \eta $ непрерывная. Тогда $ \xi + \eta $ непрерывна.
\end{remrk}
\begin{proof}[\normalfont\textsc{Доказательство}]
  \begin{align*}
		P_{\xi+\eta}(\left\{ a \right\}) = P_\xi \ast P_\eta(\left\{ a \right\}) = \int\limits_{\R} \underbrace{P_{\eta}(\left\{ a \right\} - x)}_{= 0 \text{ по непр. } \eta} dP_\xi(x) = 0.
   \end{align*}
  \end{proof}
\begin{remrk}
 Если $ U \subset \R $ --- открытое множество, и $ D $  --- не более, чем счётное множество, то
 \begin{align*}
  U = \bigsqcup_{n=1}^{\infty} \left(a_n, b_n\right],
 \end{align*} причём $ a_n, b_n \notin D $.
\end{remrk}
\begin{proof}[\normalfont\textsc{Доказательство}]
 Берём ячейки с шагом $ 1 $. Какие-то попали целиком, а какие-то не влезли. Потом берём шаг $ 1 / 2 $ и так далее.

 Если заботимся о концах: делим не совсем пополам, а примерно (так чтобы точка деления не лежала в $ D $).
\end{proof}

\begin{remrk}
	\label{remrk:lim_f_n_using_f_n_a_minnus_f_n_b}
 Пусть $ F_n $ и $ F $ --- функции распределения такие, что для любых $ a, b \in \R $
 \begin{align*}
  \lim_{n \to \infty} (F_n(b) - F_n(a)) = F(b)  - F(a).
 \end{align*} Тогда $ \forall x \in \R \colon\; \lim_{n \to \infty}  F_n(x) = F(x)$.
\end{remrk}
\begin{proof}[\normalfont\textsc{Доказательство}]
 Нужна лишь ограниченность и существование предела на бесконечностях.
 \begin{align*}
  \lim_{a \to -\infty}  F(a) = 0 \implies \exists a \in \R \colon\;  F(a) < \eps.
 \end{align*} Аналогично
 \begin{align*}
  \lim_{b \to +\infty} F(b) = 1 \implies \exists b \in \R \colon\; F(b) > 1 - \eps. 
 \end{align*} Тогда $ F(b) - F(a) > 1 -2\eps $. Тогда существует $ N $ такое, что $ \forall n > N \colon\;  $
 \begin{align*}
  \left|(F_n(b) - F_n(a)) - (F(b)-F(a)) \right| < \eps \implies F_n(b) - F_n(a) > 1 - 3\eps. 
 \end{align*} Тогда $ F_n(a) < 3\eps $ при $ n > N $. Получаем, что для любого $ x $ существует $ N' $:
 \begin{align*}
  \forall n > N' \colon\; \left|(F_n(x) - F_n(a)) - (F(x)-F(a)) \right| < \eps
 \end{align*} Следовательно,
 \begin{align*}
  \left|F_n(x) - F(x) \right| \leqslant \left| (F_n(x) - F_n(a)) - (F(x) - F(a)) \right| + F_n(a) + F(a) < 6\eps.
 \end{align*} Наверное можно оптимальнее.
\end{proof}

\begin{df}
 Множество $ B \subset \R $ \textit{регулярно} относительно распределения $ P_\xi $, если $ P_\xi(\mathrm{Cl}\,B \setminus \mathrm{Int}\,B) = 0 $.
\end{df}

В этом случае $ P(\xi \in A) = P(\xi \in B) $, если $ \mathrm{Int}\,B \subset A \subset \mathrm{Cl}\,B $.

\begin{prop*}
 У функции распределения не более, чем счётно много точек разрыва.
\end{prop*}

\begin{thm}
 Пусть $ \xi, \xi_1, \xi_2, \ldots $ --- последовательность случайных величин. $ F, F_1, F_2, \ldots $ --- их функции распределения, а $ \varphi, \varphi_1, \varphi_2, \ldots $ --- их характеристические функции. Тогда следующие условия равносильны.
 \begin{enumerate}
  \item $ \xi_n $ сходится к $ \xi $ по распределению.
  \item Для любого открытого множества $ U \subset \R $
   \begin{align*}
    \varliminf_{n \to \infty} P(\xi_n \in U) \geqslant P(\xi \in U).
   \end{align*}
  \item Для любого замкнутого множества $ A \subset \R $
   \begin{align*}
    \varlimsup_{n \to \infty} P(\xi_n \in A) \leqslant P(\xi \in A).
   \end{align*}
  \item Для любого регулярного (относительно $ P_\xi $) борелевского множества $ B \subset \R$
   \begin{align*}
    \lim_{n \to \infty} P(\xi_n \in B) = P(\xi \in B).
   \end{align*}
  \item Для любого регулярного (относительно $ P_\xi $) борелевского множества $ B $
   \begin{align*}
    \lim_{n \to \infty} \E \Ind_B(\xi_n) = \E \Ind_B(\xi).
   \end{align*}
  \item Для любой непрерывной ограниченной функции $ f \in C(\R) $
   \begin{align*}
    \lim_{n \to \infty} \E f(\xi_n) = \E f(\xi).
   \end{align*}
  \item $ \varphi_n $ сходится к $ \varphi $ поточечно.
 \end{enumerate}
\end{thm}
\begin{proof}[\normalfont\textsc{Доказательство}]
 $ 2 \iff 3 $: нужно перейти к дополнению: $ A = \R \setminus U $. Тогда
 \begin{align*}
  P(\xi_n \in A) = 1 - P(\xi_n \in U).
 \end{align*} Тогда
 \begin{align*}
	 \underbrace{\varlimsup_{n \to \infty} P(\xi_n \in A)}_{\leqslant P(\xi \in A)} = \underbrace{1 - \varliminf_{n \to \infty} P(\xi_n \in U)}_{\leqslant 1 - P(\xi \in U)}.
 \end{align*} А также:
 \begin{align*}
  P(\xi \in A) = 1 - P(\xi \in U)
 \end{align*} Получили что надо.

 $ 2 + 3 \implies 4 $. Возьмём $ A = \mathrm{Cl}\,B $, $ U = \mathrm{Int}\,B $. Тогда
 \begin{align*}
  P(\xi_n \in U) \leqslant P(\xi_n \in B) \leqslant P(\xi_n \in A).
 \end{align*} Пририсуем пределы:
 \begin{align*}
  P(\xi \in U) &\leqslant \varliminf_{n \to \infty} P(\xi_n \in U) \leqslant \varliminf_{n \to \infty} P(\xi_n \in B) \leqslant \\
  &\leqslant \varlimsup_{n \to \infty} P(\xi_n \in B) \leqslant \varlimsup_{n \to \infty} P(\xi_n \in A) \leqslant P(\xi \in A).
 \end{align*} Но так как множество $ B $ регулярное, то $ P(\xi \in U) = P(\xi \in A) = P(\xi \in B) $. Значит, это всё равенство, верхний предел равен нижнему, и он равен $ P(\xi \in B) $.

 $ 4 \iff 5 $.
 \begin{align*}
  \E \Ind_B (\xi) = P(\xi \in B), && \E \Ind_B(\xi_n) = P(\xi_n \in B)
 \end{align*} Поэтому, $ 4 $ и $ 5 $ --- это просто одно и то же, просто по-разному записанное.

 $ 6 \implies 7 $.
 \begin{align*}
  \varphi_n(t) = \E e^{it\xi_n} = \E \cos(t\xi_n) + i \E \sin(t\xi_n) \to \E\cos(t\xi) + i\E\sin(t\xi) = \E e^{it\xi} = \varphi(t).
 \end{align*} $ \cos $ и $ \sin $ ---  те самые непрерывные функции.

 Осталось три сложные стрелки.

 $ 1 \implies 2 $.
 \begin{align*}
  F_n(x) \to F(x) \quad \forall x \in D,
 \end{align*} где $ D $ -- точки разрыва $ F $. $ D $ --- не более чем, счётное множество. Представим
 \begin{align*}
  U = \bigsqcup_{n=1}^{\infty} \left(a_n, b_n\right],
 \end{align*} где $ a_n, b_n \notin D $.

 Тогда
 \begin{align*}
  P(\xi_n \in U) \geqslant P(\xi_n \in \bigsqcup_{k=1}^{m} \left(a_k, b_k\right]  ) = \sum_{k=1}^{m}P(\xi_n \in \left(a_k, b_k\right]  ) = \sum_{k=1}^{m}(F_n(b_k) - F_n(a_k)) \to \\
  \to \sum_{k=1}^{m}(F(b_k)-F(a_k)) = P(\xi \in \bigsqcup_{k=1}^{m}\left(a_k, b_k\right]  )
 \end{align*} Пририсуем пределы:
 \begin{align*}
  \varliminf_{n \to \infty} P(\xi_n \in U) \geqslant \varliminf_{n \to \infty} P(\xi_n \in \bigsqcup_{k=1}^{m}\left(a_k, b_k\right]  ) = P(\xi \in \bigsqcup_{k=1}^{m}\left(a_k, b_k\right]  ).
 \end{align*} Значит
 \begin{align*}
  \varliminf_{n \to \infty} P(\xi_n \in U) \geqslant \varliminf_{m \to \infty} P(\xi \in \bigsqcup_{k=1}^{m} \left(a_k, b_k\right]  ) = P(\xi \in U),
 \end{align*} по непрерывности меры сверху.

 $ 5 \implies 6 $. Возьмём $ D = \left\{ x \in \R \mid P(f(\xi) = x) > 0 \right\} $. Множество $ D $ не более, чем счётно, так как сумма
 \begin{align*}
  \sum_{x \in D} P(f(\xi) = x) < 1
 \end{align*} конечна.  Функция $ f $ ограничена: $ \left| f \right| \leqslant M $, причём $ M \notin D $(если все таки попало в $D$ то слегка увеличим $M$). Разобьём отрезок $ [-M,M] $ на много примерно одинаковых отрезочков: разрезы по точкам $ -M = t_0, t_1, t_2, \ldots, t_{m-1}, t_m = M $, причём
 \begin{align*}
  t_j - t_{j-1} < \frac{3M}{m} \text{ и } t_j \not \in D
\end{align*} --- разрезы примерно равные, см. рисунок \eqref{fig:7_points_theorem_cut_m} .
\begin{figure}[ht]
    \centering
	\incfig[0.5]{7_points_theorem_cut_m}
	\caption{Разрезание отрезка $[-M, M]$}
    \label{fig:7_points_theorem_cut_m}
\end{figure}
 Рассмотрим множества
 \begin{align*}
  B_j = \left\{ x \in \R \mid t_{j-1} < f(x) \leqslant t_j \right\},
 \end{align*} $ U_j \subset B_j \subset A_j $где $ U_j = \left\{ x \in \R \mid t_{j-1} < f(x) < t_j \right\} $ и $ A_j = \left\{ t_{j-1} \leqslant f(x) \leqslant t_j \right\} $. Но $ U_j $ открыто (прообраз открытого), а $ A_j $ замкнутое (прообраз замкнутого). Тогда
 \begin{align*}
  U_j \subset \mathrm{Int}\, B_j \subset B_j \subset \mathrm{Cl}\,  B_j \subset A_j.
 \end{align*} Запишем
 \begin{align*}
  P_\xi(\mathrm{Cl}\,B_j \setminus \mathrm{Int}\,B_j) \leqslant P_\xi(A_j \setminus U_j) = P(f(\xi) \in \left\{ t_{j-1}, t_j \right\}) \underbrace{=}_{\text{т.к. $t_j, t_{j - 1} \not \in D$ }} 0.
 \end{align*} Значит, все множества $ B_j $ регулярные. Тогда по пункту 5:
 \begin{align*}
  \E\Ind_{B_j}(\xi_n) \to \E\Ind_{B_j}(\xi).
 \end{align*} Заведём функцию
 \begin{align*}
  g(x) = \sum_{k=1}^{m} t_{j-1} \Ind_{B_j}(x).
 \end{align*} Тогда $ g(x) \leqslant f(x) \leqslant g(x) + \text{мелкость} < g(x) + \eps $(где $\eps$ равно $\frac{3M}{m}$  ). Тогда
 \begin{align*}
  \left| g(x) - f(x) \right| < \eps
 \end{align*} для всех $ x $, и тогда $ \E \left| f(\xi_n)-g(\xi_n) \right|< \eps $, $ \E \left| f(\xi)-g(\xi) \right| < \eps $. Ещё мы знаем, что
 \begin{align*}
  \E g(\xi_n) \to \E g(\xi),
 \end{align*} так как $ g $ --- линейная комбинация $ \Ind_{B_j} $. Теперь $ 3\eps $-приём:
 \begin{align*}
  \left| \E f(\xi_n) - \E f(\xi) \right| \leqslant \underbrace{\left| \E f(\xi_n) - \E g(\xi_n) \right|}_{<\eps} + \underbrace{\left| \E g(\xi_n) - \E g(\xi) \right|}_{\to 0} + \underbrace{\left| \E g(\xi) - \E f(\xi) \right|}_{< \eps} < 3\eps
 \end{align*} при больших $ n $.

 $ 7 \implies 1 $. Воспользуемся формулой обращения:
 \begin{align*}
  P(\xi \in (a,b]) = \lim_{T \to +\infty}  \frac{1}{2\pi} \int\limits_{-T}^{T} \frac{e^{-iat} - e^{-ibt}}{it} \varphi_\xi(t)\,dt,
 \end{align*} если $ P(\xi = a) = P(\xi = b) = 0 $. Здесь проблема в том, что интеграл не настоящий, а в смысле главного значения. Решать проблему будем так: изменим случайную величину, так чтобы не сильно поменялись значения, но характеристическая функция улучшилась бы. Возьмём случайную величину $ \eta $, не зависящую от всех $ \xi, \xi_1, \xi_2, \ldots $ такую, что $ \eta \sim \Norm(0, \sigma^{2}) $ (формально это очень занудно). Тогда мы знаем, что
 \begin{align*}
  \varphi_{\xi_n + \eta}(t) = \varphi_{\xi_n}(t) \cdot \varphi_\eta(t) = \varphi_{\xi_n}(t) e^{-\sigma^{2}t^{2} / 2} \to \varphi_\xi(t) e^{-\sigma^{2}t^{2} / 2} = \varphi_{\xi+\eta}(t)
 \end{align*} поточечно. Кроме того, случайные величины $ \xi_n + \eta $ и   $ \xi + \eta $ непрерывны. Поэтому не нужно оговорка. Тогда
   \begin{align*}
		 P(\xi_n + \eta \in [a,b]) = \lim_{T \to +\infty}  \frac{1}{2\pi} \int\limits_{-T}^T \underbrace{\frac{e^{-iat} - e^{-ibt}}{it}}_{|\cdot| \leqslant const} \underbrace{\varphi_{\xi_n}(t)}_{|\cdot| \leqslant 1} e^{-\sigma^{2}t^{2} / 2} \,dt
	 \end{align*} Но интеграл сходится, так как $\int_{-\infty}^{+\infty} e^{-\sigma^2t^2/2}\,dt$ сходится, а остальные сомножители под интегралом ограничены! Поэтому можно убрать предел:
 \begin{align*}
	 = \frac{1}{2\pi} \int\limits_{\R} \frac{e^{-iat} - e^{-ibt}}{it} e^{-\sigma^{2}t^{2} / 2}\varphi_{\xi_n}(t)\,dt
 \end{align*} Устремим $ n \to \infty $. По теореме Лебега с мажорантой $const \cdot e^{\sigma^2t^2/2}$ переходим к пределу под интегралом:  
 \begin{align*}
	 \to \frac{1}{2\pi} \int\limits_{\R} \frac{e^{-iat}-e^{-ibt}}{it}  e^{-\sigma^{2}t^{2} / 2}\varphi_{\xi}(t)\,dt
 \end{align*} Но
 \begin{align*}
  = P(\xi+ \eta \in (a,b]).
 \end{align*} Пусть $ G_n $ и $ G $ --- функции распределения для $ \xi_n + \eta $ и $ \xi + \eta $ соответственно. Тогда мы только что показали, что
 \begin{align*}
  G_n(b) - G_n(a) \to G(b) - G(a)
 \end{align*} для любых $ a,b\in\R $. По замечанию \ref{remrk:lim_f_n_using_f_n_a_minnus_f_n_b} $ G_n(x) \to G(x) $ для любого $ x \in \R $.

 Осталось перейти от $ G $ к $ F $. Пусть $ x $ --- точка непрерывности $ F $. Зафиксируем $ \eps  > 0 $. Выберем $ \delta > 0 $ так, что 
 \begin{align*}
  \left| F(x \pm 2\delta) - F(x) \right| < \eps.
 \end{align*}
 \begin{align*}
  \left\{ \xi+\eta \leqslant x  + \delta \right\} \subset \left\{ \xi \leqslant x + 2\delta \right\} \cup \left\{ \left| \eta \right| \geqslant \delta \right\}.
 \end{align*} Запишем неравенство на вероятности:
 \begin{align*}
	 G(x+\delta) \leqslant F(x+2\delta) + P(\left| \eta \right| \geqslant \delta) \leqslant F(x+2\delta) + \frac{\D \eta}{\delta^{2}} = F(x + 2\delta) + \frac{\sigma^2}{\delta^2}. 
 \end{align*} 
 Тогда при больших $n$
 \begin{align*}
	 F_n(x) = P(\xi_n < x) &\leqslant P(\xi_n + \eta \leqslant x + \delta) + P(|\eta| \geqslant \delta) \\
												 &\leqslant G_n(x+\delta) + \frac{\sigma^{2}}{\delta^{2}} \\
												 &< G(x+\delta) + \eps + \frac{\sigma^{2}}{\delta^{2}} \\
												 &\leqslant F(x+2\delta) + \eps + \frac{2\sigma^{2}}{\delta^{2}} \\
												 &< F(x) + 2\eps + \frac{2\sigma^{2}}{\delta^{2}}.
 \end{align*}

 Напишем похожую штуку в другую сторону:
 \begin{align*}
  \left\{ \xi +\eta \leqslant x - \delta \right\} \supset \left\{ \xi \leqslant x - 2\delta \right\} \setminus \left\{ \left| \eta \right| \geqslant \delta \right\}.
 \end{align*} Для вероятностей:
 \begin{align*}
  G(x-\delta) \geqslant F(x - 2\delta) - \frac{\sigma^{2}}{\delta^{2}}.
 \end{align*}  Аналогично получаем оценку снизу 
 \begin{align*}
  F_n(x) \geqslant G_n(x - \delta) - \frac{\sigma^{2}}{\delta^{2}} > G(x-\delta) - \eps - \frac{\sigma^{2}}{\delta^{2}} \geqslant F(x - 2\delta) - \eps - \frac{2\sigma^{2}}{\delta^{2}}  > F(x) - 2\eps - \frac{2\sigma^{2}}{\delta^{2}}.
 \end{align*}

 Тогда
 \begin{align*}
  \left| F_n(x) - F(x) \right| < 2\eps + 2 \frac{\sigma^{2}}{\delta^{2}}
 \end{align*} Выбираем $ \eps > 0$, по $ \eps $ выбираем $ \delta $,  а затем по $ \delta $ выбираем $ \sigma $, и получаем нужное $ N $. Так и построим $ N $ по $ \eps $.

\end{proof}

\begin{thm}
 Пусть функции $ F_n, F \colon\, \R \to [0,1] $(они возрастают), $ F \in C(\R) $ и $ F_n \to F $ поточечно. Тогда $ F_n \rightrightarrows F $ равномерно.
\end{thm}
\begin{proof}[\normalfont\textsc{Доказательство}]
 Найдём $ t_j $ такое, что $ F(t_j) = \frac{j}{m} $, где $ j = 1, 2, \ldots, m - 1 $. Мы знаем, что для всех $ j $ верно $ F_n(t_j) \to F(t_j) $, поэтому при больших $ n $ верно
 \begin{align*}
  \left| F_n(t_j) - F(t_j) \right| < \eps
 \end{align*}

\begin{figure}[ht]
    \centering
	\incfig[0.5]{distribution_function_points_to_evenlly}
	\caption{Выбор $t_1, \dots, t_{m-1}$. }
    \label{fig:distribution_function_points_to_evenlly}
\end{figure}

 Тогда возьмём $ t_{j-1} < t \leqslant t_j $. Тогда при больших $ n $:
 \begin{align*}
  F_n(t) \leqslant F_n(t_j) < F(t_j) + \eps = \frac{j}{m} + \eps = F(t_{j-1}) + \frac{1}{m} + \eps \leqslant F(t) + \eps + \frac{1}{m}.
 \end{align*} В другую сторону аналогично:
 \begin{align*}
  F_n(t) \geqslant F_n(t_{j-1}) > F(t_{j-1}) - \eps = \frac{j-1}{m} - \eps = F(t_j) - \frac{1}{m} - \eps \geqslant F(t) - \eps - \frac{1}{m}.
 \end{align*} Значит берём $ \frac{1}{m} < \eps $ и успех.
\end{proof}

\section{Центральная предельная теорема.}

\begin{thm}[ЦПТ в форме Поля Лев\'{и}]
 Пусть случайные величины $ \xi_1, \xi_2, \ldots $  одинаково распределены и независимы. Пусть $ a = \E \xi_i $ --- их матожидание, и  $ \sigma^{2} = \D \xi_i > 0 $  --- их дисперсия,
 \begin{align*}
  S_n = \xi_1 + \xi_2 + \ldots + \xi_n
 \end{align*} --- их частичная сумма. Тогда
 \begin{align*}
  P\left(\frac{S_n - \E S_n}{\sqrt{\D S_n}} \leqslant x\right) = P \left( \frac{S_n - na}{\sigma \sqrt n} \leqslant x \right) \rightrightarrows \Phi(x),
 \end{align*} где
 \begin{align*}
  \Phi(x) = \frac{1}{\sqrt{2\pi}} \int\limits_{-\infty}^{x} e^{-t^{2} / 2}\,dt.
 \end{align*}
\end{thm}
\begin{remrk*}
 Поль Лев\'{и}(1886 - 1971) --- не тот, что и Л\'{е}ви(Беппо Леви, 1875 - 1961) из теоремы в теории меры о перестановке знака предела и интеграла Лебега.
\end{remrk*}
\begin{proof}[\normalfont\textsc{Доказательство}]
 Разложим
 \begin{align*}
  \varphi(t) = \varphi_{\xi_1 - a}(t) = 1 - \frac{\sigma^{2}t^{2}}{2} + o(t^{2})
 \end{align*} Положим
 \begin{align*}
  \varphi_n(t) &:= \varphi_{\frac{S_n - na}{\sigma \sqrt n}}(t) = \varphi_{S_n - na} \left( \frac{t}{\sigma \sqrt n} \right) = \prod_{k=1}^{n} \varphi_{\xi_k - a} \left( \frac{t}{\sigma \sqrt n} \right) = \\
  &= \left(\varphi \left( \frac{t}{\sigma \sqrt n} \right) \right)^{n} = \left( 1- \frac{\sigma^{2} \frac{t^{2}}{\sigma^{2} n}}{2} + o \left( \frac{t^{2}}{\sigma^{2} n} \right) \right)^{n} = \\
  &= \left( 1 - \frac{t^{2}}{2n} + o \left( \frac{t^{2}}{n} \right) \right)^{n} \to e^{-t^{2} / 2}.
 \end{align*} Осталось проверить поточечную сходимость характеристических функций:

 \begin{align*}
	 \left(1 - \frac{t^2}{2n} + o\left(\frac{t^2}{n}\right)\right)^{n} \to e^{-t^2/2} \Longleftrightarrow \\
	 n \log \left(1 - \frac{t^2}{2n} + o\left(\frac{t^2}{n}\right)\right) \to - \frac{t^2}{2} \Longleftrightarrow \\
	 n \left(-\frac{t^2}{2n} + o\left(\frac{t^2}{n}\right)\right) \to -\frac{t^2}{2} \Longleftrightarrow \\
	 -\frac{t^2}{2} \to -\frac{t^2}{2}
 \end{align*}. Получили что и требовалось.

\end{proof}

\begin{crly}[Интегральная теорема Муавра-Лапласа]
 Пусть $ S_n  $ --- количество успехов в схеме Бернулли в схеме с  $ n $  испытаниями и вероятностью успеха $ p $ . Тогда
 \begin{align*}
  P \left( \frac{S_n - np}{\sqrt{npq}} \leqslant x \right) \rightrightarrows \Phi(x).
 \end{align*}
\end{crly}
\begin{proof}[\normalfont\textsc{Доказательство}]
 Пусть
 \begin{align*}
  \xi_k = \begin{cases}
   1, \text{ с вероятностью } p, \\
   0, \text{ с вероятностью } q = 1 - p.
  \end{cases}
 \end{align*} Тогда $ \E \xi_k = p $, $ \D \xi_k = pq $, $ \xi_k $ независимы. Применим ЦПТ.
\end{proof}

\begin{thm}[Пуассона]
 Пусть $ P(\xi_{nk} = 1) = p_{nk} $, $ P(\xi_{nk} = 0) = 1-p_{nk} $. Много бернуллевских случайных величин. Пачка $ \xi_{n 1}, \xi_{n 2}, \ldots, \xi_{n n} $ независимы. Пусть
 \begin{align*}
  \max_{k=1}^{n} p_{nk} \to 0
 \end{align*} при $ n \to \infty $. Пусть также
 \begin{align*}
  p_{n 1}  + p_{n 2} + \ldots + p_{n n} \to \lambda > 0.
 \end{align*} Пусть $ S_n = \xi_{n 1} + \xi_{n 2} + \ldots + \xi_{n n} $. Тогда
 \begin{align*}
  P(S_n = k) \to \frac{e^{-\lambda}\lambda^{k}}{k!}.
 \end{align*}
\end{thm}
В стандартной теореме Пуассона $ p_{n 1} = p_{n 2 } = \ldots = p_{n n} $.
\begin{proof}[\normalfont\textsc{Доказательство}]
 Посмотрим на характеристические функции $ \xi_{n k} $:
 \begin{align*}
  \varphi_{\xi_{n k}}(t) = \E e^{it \xi_{n k}} = (1 - p_{n k}) + e^{it} p_{n k} = 1 + (e^{it} - 1)p_{n k},
 \end{align*} ведь случайная величина принимает лишь два значения.
 \begin{align*}
  \varphi_{S_n}(t) = \prod_{k=1}^{n} \varphi_{\xi_{n k}}(t) = \prod_{k=1}^{n} (1 + (e^{it} - 1)p_{n k}).
 \end{align*} Запишем характеристическую функцию Пуассона:
 \begin{align*}
  \exp((e^{it} - 1)\lambda).
 \end{align*} Проверим: прологарифмируем:
 \begin{align*}
  \log \varphi_{S_n}(t) = \sum_{k=1}^{n}\log (1 + (e^{it} - 1)p_{nk}) \overset{?}{\to} (e^{it} - 1) \lambda
 \end{align*} Но
 \begin{align*}
  \log(1 + (e^{it} - 1)p_{n k}) = (e^{it} - 1)p_{n k} + O(p_{n k}^{2})
 \end{align*} Тогда сумма
 \begin{align*}
	 = \sum_{k=1}^{n} \left( (e^{it}-1)p_{nk} +O(p_{nk}^{2}) \right) = (e^{it} - 1) \underbrace{\sum_{k=1}^{n}p_{nk}}_{\to \lambda} + \sum_{k = 1}^{n} O(p_{nk} ^{ 2}).
 \end{align*} Оценим остаток:
 \begin{align*}
	 \sum_{k=1}^{n}p_{nk}^{2} \leqslant \sum_{k=1}^{n} p_{nk} \cdot \max p_{n k} \to \lambda \cdot \underbrace{\max p_{nk}}_{\to 0} \to 0.
 \end{align*} Мы поняли, что $ \varphi_{S_n}(t) \to \exp((e^{it}-1)\lambda) $. Тогда есть сходимость по распределению: $ S_n $ сходится по распределению к Пуассону. Так как всё дискретно, то условие <<точки непрерывности>> --- не проблема.
\end{proof}
\begin{remrk*}
 Вычислим характеристическую функцию распределения Пуассона:
 \begin{align*}
	 \varphi(t) &= \E e^{it\xi} = \sum_{n=0}^{\infty} e^{itn} P(\xi = n) = \sum_{n=0}^{\infty} (e^{it})^{n} \cdot \frac{e^{-\lambda} \lambda^{n}}{n!} = e^{-\lambda} \sum_{n = 0}^{\infty} \frac{(e^{it}\lambda)^{n}}{n!} = \\
	 &= e^{-\lambda} \cdot e^{\lambda e^{it}} = \exp((e^{it} - 1)\lambda).
 \end{align*}
\end{remrk*}

\begin{thm}[%
ЦПТ в форме Линденберга]
 Пусть случайные величины $ \xi_1, \xi_2, \ldots $  независимы, $ a_k = \E \xi_k $,  $ \sigma_k^{2} = \D \xi_k > 0 $ . Пусть $ S_n = \xi_1 + \ldots + \xi_n $, и пусть
 \begin{align*}
  DS_n = D_n^{2} = \sum_{k=1}^{n} \sigma_k^{2}.
 \end{align*} Рассмотрим функцию $ f(x) := x^{2} \Ind_{\left\{ \left| x \right| \geqslant \eps D_n \right\}} $. Обозначим
 \begin{align*}
  \mathrm{Lind}(\eps, n) := \frac{1}{D_n^{2}} \sum_{k=1}^{n} \E f(\xi_k - a_k).
 \end{align*} Пусть для любого $ \eps > 0 $ верно
  \begin{align*}
  \lim_{n \to \infty} \mathrm{Lind}(\eps, n) = 0
 \end{align*} --- \textit{условие Линденберга}. Тогда верно заключение ЦПТ:
 \begin{align*}
  P \left( \frac{S_n - \E S_n}{\sqrt{\D S_n}} \leqslant x \right) \rightrightarrows \Phi(x).
 \end{align*}
\end{thm}

Доказывать не будем (много считать).

\begin{exercs*}
 Проверить условие Линденберга для независимых одинаково распределённых случайных величин с конечной дисперсией.
\end{exercs*}

\begin{thm}[%
ЦПТ в форме Ляпунова]
 Пусть $ \xi_1, \xi_2, \ldots $ независимы. $ a_k = \E \xi_k $, $ \sigma_k^{2} = \D \xi_k > 0 $, пусть
 \begin{align*}
  L(\delta, n) = \frac{1}{D_n^{2+\delta}} \sum_{k=1}^{n} \E \left| \xi_k - a_k \right|^{2 + \delta} \to 0
 \end{align*}, где $ D_n $ и $ S_n $ --- то же самое. для некоторого $ \delta  > 0 $. Тогда верно заключение ЦПТ:
 \begin{align*}
  P \left( \frac{S_n - \E S_n}{\sqrt{\D S_n}} \leqslant x \right) \rightrightarrows \Phi(x).
 \end{align*}
\end{thm}
\begin{proof}[\normalfont\textsc{Вывод из ЦПТ в форме Линденберга}]
 \begin{align*}
  \mathrm{Lind}(\eps, n) &= \frac{1}{D_n^{2}} \sum_{k=1}^{n} \E \left( (x_k-a_k)^{2}\Ind_{\left\{ \left| x \right| \geqslant \eps D_n\right\}}(\xi_k - a_k) \right).
 \end{align*} Заметим, что $ \Ind_{\left\{ \left| x \right| \geqslant \eps D_n \right\}} \leqslant \left(\frac{\left| x \right|}{\eps D_n} \right)^{\delta} $. Подставим вместо индикатора эту дробь:
 \begin{align*}
    \leqslant \frac{1}{D_n^{2}} \sum_{k=1}^{n} \E \left( (\xi_k-a_k)^{2} \cdot \frac{\left| \xi_k-a_k \right|^{\delta}}{\eps^{\delta} D_n^{\delta}} \right) = \frac{L(\delta, n)}{\eps^{\delta}}.
 \end{align*} 
\end{proof}
\begin{thm}[Оценка для Ляпунова]
 \begin{align*}
  \left| P \left( \frac{S_n-\E S_n}{\sqrt{\D S_n}} \leqslant x\right) - \Phi(x) \right| \leqslant C_\delta L(\delta, n),
 \end{align*} где $ C_\delta $ --- константа, зависящая от $ \delta $, для любого $ \delta \in (0, 1] $.
\end{thm}
\begin{crly}
 Если $ \xi_1, \xi_2, \ldots $  одинаково распределены и независимы, то
 \begin{align*}
  \left| P \left( \frac{S_n - na}{\sigma \sqrt n} \leqslant x \right) - \Phi(x) \right| \leqslant C_\delta,
 \end{align*} $ D_n^{2} = n\sigma^{2} $. Тогда
 \begin{align*}
  L(\delta, n) = \frac{1}{n^{1 + \frac{\delta}{2}}} \cdot \frac{1}{\sigma^{2 + \delta}} n \E \left| \xi_1 - a \right|^{2 + \delta}
 \end{align*}
 \begin{align*}
  \leqslant C_\delta \frac{\E \left| \xi_1 - a \right|^{2+\delta}}{n^{\delta / 2} \sigma^{2 + \delta}}.
 \end{align*}
\end{crly}

\begin{thm}[Берри-Эссеена]
 Пусть $ \xi_1, \xi_2, \ldots $ независимы и одинаково распределены.  Тогда
\begin{align*}
\left| P \left( \frac{S_n - na}{\sigma \sqrt n} \leqslant x \right) - \Phi(x) \right| \leqslant C \frac{\E \left| \xi_1 - a \right|^{3}}{\sqrt n \cdot \sigma^{3}}.
\end{align*}
\end{thm}

Что известно про константы?

\begin{itemize}
 \item Эссеен (1956): $ C \geqslant \frac{3 + \sqrt{10}}{6 \sqrt{2\pi}} \approx 0.4097 $.
 \item Шевцова (2014): $ C \leqslant 0.469 $.
\end{itemize}

Для общего случая известно $ C_1 \leqslant 0.5583 $.

Для схемы Бернулли $ C \leqslant 0.4099 $ (не хватает $ 0.002 $ до нижней оценки!) (2018)

Для схемы Бернулли  с $ p = \frac{1}{2} $: $ C = \frac{1}{\sqrt{2\pi}} $.

Чисто для справки приведём ещё два результата.

\begin{thm}[Хартмана-Винтнера, закон повторного логарифма]
 Пусть случайные величины $ \xi_1, \ldots $ независимы и одинаково распределены, $ \E \xi_i = 0 $, $ \sigma^{2} = \D \xi_i > 0 $. Тогда
 \begin{align*}
  \varlimsup_{n \to \infty} \frac{S_n}{\sqrt{2n \log\log n}} = \sigma,
 \end{align*} и
 \begin{align*}
  \varliminf_{n \to \infty} \frac{S_n}{\sqrt{2n \log\log n}} = -\sigma.
 \end{align*}
\end{thm}

\begin{thm}[Штрассена]
 Любая точка из $ [-\sigma, \sigma] $ является предельной точкой последовательности
  \begin{align*}
   \frac{S_n}{\sqrt{2 n \log \log n}}.
 \end{align*}
\end{thm}

\end{document}
